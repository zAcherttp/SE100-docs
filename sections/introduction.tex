\documentclass[../main.tex]{subfiles}
\begin{document}

\chapter{Giới thiệu}

\section{Mục đích}
Tài liệu đặc tả yêu cầu phần mềm (SRS) đóng vai trò như một phương tiện giúp truyền đạt thông tin một cách rõ ràng và nhất quán giữa các bên tham gia dự án. Tài liệu này được trình bày bằng ngôn ngữ phổ thông, dễ hiểu, giúp mọi thành viên trong tổ chức — dù có nền tảng kỹ thuật hay không — đều có thể tiếp cận và nắm bắt nội dung.

Tài liệu được biên soạn dành cho các bên liên quan và đội ngũ phát triển phần mềm, nhằm cung cấp cái nhìn toàn diện về yêu cầu của hệ thống. Ngoài các yêu cầu chức năng, tài liệu còn đề cập đến các yếu tố quan trọng khác như độ tin cậy, an toàn bảo mật, và khả năng bảo trì của hệ thống.

\section{Phạm vi sản phẩm}
Hệ thống có tên là Car Garage Management System. Đây là một ứng dụng nội bộ dựa trên web được thiết kế nhằm hỗ trợ quản lý toàn bộ hoạt động của gara ô tô, bao gồm tiếp nhận xe, phân công sửa chữa, theo dõi tiến độ, tính toán chi phí và cho khách hàng theo dõi.

Hệ thống bao gồm ba loại người dùng chính: quản trị viên (administrator), kỹ thuật viên (mechanic) và khách hàng (customer). Mỗi nhóm người dùng có những vai trò và chức năng riêng biệt trong hệ thống, được phân quyền rõ ràng để đảm bảo hoạt động hiệu quả và an toàn.

Mục tiêu của hệ thống là tự động hóa các quy trình thủ công trong gara, giúp giảm thiểu sai sót, tiết kiệm thời gian và nâng cao chất lượng dịch vụ. Cụ thể, quản trị viên có thể quản lý toàn bộ thông tin gara, nhân viên và báo cáo tổng hợp chỉ với vài thao tác đơn giản.Kỹ thuật viên có thể xem danh sách công việc, cập nhật trạng thái sửa chữa của từng xe và báo cáo tiến độ trực tiếp lên hệ thống. Trong khi đó, khách hàng có thể đăng nhập để kiểm tra thông tin xe, tình trạng sửa chữa, và chi phí dự kiến mà không cần liên hệ trực tiếp với nhân viên.

Ví dụ, khi một khách hàng đặt lịch bảo dưỡng trực tuyến, hệ thống sẽ tự động cập nhật thông tin vào lịch làm việc của kỹ thuật viên và danh sách xe chờ xử lý. Nhờ đó, nhân viên có thể chuẩn bị trước phụ tùng, thiết bị cần thiết, góp phần nâng cao hiệu quả và độ chuyên nghiệp của gara.

\section{Đối tượng sử dụng và tổng quan tài liệu}
Tài liệu này đóng vai trò là hướng dẫn ban đầu cho các nhà phát triển, quản lý dự án, kiểm thử viên và người dùng có liên quan trực tiếp đến việc sử dụng hệ thống Car Garage Management System. Nội dung của tài liệu bao gồm yêu cầu phần cứng, phần mềm và hướng dẫn sử dụng hệ thống.

Các nhà phát triển (developers) khi muốn đọc, chỉnh sửa hoặc bổ sung các yêu cầu mới cho hệ thống nên tham khảo tài liệu này trước để đảm bảo sự nhất quán và đúng định hướng.

Người dùng (users) có thể xem các sơ đồ và bản mô tả chức năng trong tài liệu để đánh giá xem phần mềm có đáp ứng đầy đủ các yêu cầu đặt ra hay không, và để hiểu rõ hơn cách hệ thống hoạt động. Trong trường hợp có điểm chưa rõ, họ có thể tham khảo tài liệu hướng dẫn sử dụng (user manual) để làm rõ. 

Đối với kiểm thử viên (testers), tài liệu này là cơ sở quan trọng để họ xây dựng bộ kiểm thử (test cases), nhằm xác minh rằng các yêu cầu ban đầu của dự án đã được triển khai và hoạt động chính xác trong sản phẩm cuối cùng.

\end{document}