\documentclass[../../main.tex]{subfiles}

\begin{document}

\begin{table}[h]
\centering
\begin{tabular}{|p{4cm}|p{10cm}|}
\hline
\textbf{Mã UC} & UC06-03 \\
\hline
\textbf{Tên UC} & Thực thi lấy hàng \\
\hline
\textbf{Mô tả tóm tắt} & Nhân viên kho thực hiện việc lấy hàng trong kho theo danh sách được giao trong phiên lấy hàng. \\
\hline
\textbf{Tác nhân tham gia} & Nhân viên kho \\
\hline
\textbf{Luồng sự kiện chính} & 
\begin{enumerate}
    \vspace{-18pt}
    \item Nhân viên mở phiên lấy hàng.
    \item Thực hiện quét mã sản phẩm và lấy hàng theo chỉ dẫn.
    \item Hệ thống xác nhận từng sản phẩm đã được lấy.
\end{enumerate} \\
\hline
\textbf{Luồng sự kiện khác} & 
\begin{enumerate}
    \vspace{-18pt}
    \item[4a.] Nếu hàng không còn trong vị trí → hiển thị cảnh báo và yêu cầu cập nhật. 
\end{enumerate} \\
\hline
\textbf{Yêu cầu đặc biệt} & Cần thiết bị quét mã vạch. \\
\hline
\textbf{Trạng thái hệ thống ngay khi bắt đầu thực hiện} & Phiên lấy hàng đang mở. \\
\hline
\textbf{Trạng thái hệ thống sau khi bắt đầu thực hiện} & Hàng hóa được xác nhận đã lấy xong. \\
\hline
\textbf{Điểm mở rộng} & Bao gồm UC06-04 Hoàn thành phiên lấy hàng. \\
\hline
\end{tabular}

\caption{Bảng UC06-03}
\end{table}

\end{document}
