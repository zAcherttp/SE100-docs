
\documentclass[../../main.tex]{subfiles}

\begin{document}

\begin{table}[h]
\centering
\caption{Đặc tả use case}
\begin{tabular}{|p{4cm}|p{10cm}|}
\hline
\textbf{Mã UC} & UC05-01 \\
\hline
\textbf{Tên UC} & Tạo phiên nhập hàng \\
\hline
\textbf{Mô tả tóm tắt} & Nhân viên nhận hàng tạo phiên nhập hàng mới, liên kết với đơn mua hàng (PO) để bắt đầu quá trình nhận hàng. \\
\hline
\textbf{Tác nhân tham gia} & Chuyên viên nhận hàng, Quản lý kho \\
\hline
\textbf{Luồng sự kiện chính} & 
\begin{enumerate}
    \vspace{-18pt}
    \item Chọn chức năng 'Tạo phiên nhập hàng'.
    \item Nhập mã PO hoặc chọn PO liên quan, nhà cung cấp, ngày dự kiến và người thực hiện.
    \item Hệ thống tạo phiên nhập và lưu trạng thái 'Đang nhập'.
    \item Hệ thống hiển thị giao diện để quét mã/số seri và nhập số lượng.
\end{enumerate} \\
\hline
\textbf{Luồng sự kiện khác} & 
\begin{enumerate}
    \vspace{-18pt}
    \item[4a.] Nếu PO không tồn tại hoặc không hợp lệ → hệ thống báo lỗi và yêu cầu nhập lại.
\end{enumerate} \\
\hline
\textbf{Yêu cầu đặc biệt} & Không có \\
\hline
\textbf{Trạng thái hệ thống ngay khi bắt đầu thực hiện} & Người dùng đã đăng nhập và có quyền tạo phiên nhập. \\
\hline
\textbf{Trạng thái hệ thống sau khi bắt đầu thực hiện} & Phiên nhập được tạo trong trạng thái 'Đang nhập'. \\
\hline
\textbf{Điểm mở rộng} & Bao gồm UC05-02 Quét mã vạch \\
\hline
\end{tabular}
\end{table}

\end{document}
