\documentclass[../../main.tex]{subfiles}

\begin{document}

\begin{table}[h]
\centering
\caption{Bảng UC03-05}
\begin{tabular}{|p{4cm}|p{10cm}|}
\hline
\textbf{Mã UC} & UC03-05 \\
\hline
\textbf{Tên UC} & Xem phân cấp khu vực \\
\hline
\textbf{Mô tả tóm tắt} & Người dùng có thể xem cấu trúc phân cấp của các khu vực lưu trữ trong kho để dễ dàng quản lý. \\
\hline
\textbf{Tác nhân tham gia} & Quản lý kho, Nhân viên kho \\
\hline
\textbf{Luồng sự kiện chính} &
\begin{enumerate}
    \vspace{-18pt}
    \item Người dùng chọn chức năng “Xem phân cấp khu vực”.
    \item Hệ thống truy xuất dữ liệu khu vực từ cơ sở dữ liệu.
    \item Hiển thị phân cấp dưới dạng cây (kho → khu → vị trí → ô chứa hàng).
\end{enumerate} \\
\hline
\textbf{Luồng sự kiện khác} &
\begin{enumerate}
    \vspace{-18pt}
    \item[4a.] Nếu không có dữ liệu khu vực → hiển thị thông báo “Chưa có khu vực được cấu hình”.
\end{enumerate} \\
\hline
\textbf{Yêu cầu đặc biệt} & Giao diện cần hỗ trợ hiển thị dạng cây và thao tác cuộn. \\
\hline
\textbf{Trạng thái hệ thống ngay khi bắt đầu thực hiện} & Các khu vực đã được cấu hình trong hệ thống. \\
\hline
\textbf{Trạng thái hệ thống sau khi bắt đầu thực hiện} & Phân cấp khu vực được hiển thị rõ ràng. \\
\hline
\textbf{Điểm mở rộng} & Bao gồm UC03-06 Trực quan hóa sơ đồ 2D. \\
\hline
\end{tabular}
\end{table}

\end{document}
