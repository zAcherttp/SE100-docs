\documentclass[../../main.tex]{subfiles}

\begin{document}

\begin{table}[h]
\centering
\caption{Bảng UC04-06}
\begin{tabular}{|p{4cm}|p{10cm}|}
\hline
\textbf{Mã UC} & UC04-06 \\
\hline
\textbf{Tên UC} & Tính điểm đặt hàng lại (Hệ thống) \\
\hline
\textbf{Mô tả tóm tắt} & Hệ thống tự động tính toán điểm đặt hàng lại (ROP) dựa trên dữ liệu tồn kho, nhu cầu và thời gian giao hàng. \\
\hline
\textbf{Tác nhân tham gia} & (Hệ thống) \\
\hline
\textbf{Luồng sự kiện chính} &
\begin{enumerate}
    \vspace{-18pt}
    \item Hệ thống định kỳ truy xuất dữ liệu tồn kho và mức tiêu thụ.
    \item Tính điểm ROP cho từng sản phẩm.
    \item Lưu kết quả và gợi ý đơn đặt hàng mới.
\end{enumerate} \\
\hline
\textbf{Luồng sự kiện khác} &
\begin{enumerate}
    \vspace{-18pt}
    \item[4a.] Nếu dữ liệu tồn kho thiếu → bỏ qua sản phẩm đó. 
\end{enumerate} \\
\hline
\textbf{Yêu cầu đặc biệt} & Yêu cầu dữ liệu tồn kho cập nhật chính xác. \\
\hline
\textbf{Trạng thái hệ thống ngay khi bắt đầu thực hiện} & Hệ thống hoạt động định kỳ. \\
\hline
\textbf{Trạng thái hệ thống sau khi bắt đầu thực hiện} & Điểm đặt hàng lại được tính và lưu trữ. \\
\hline
\textbf{Điểm mở rộng} & Bao gồm UC04-07 Tạo cảnh báo tồn kho thấp. \\
\hline
\end{tabular}
\end{table}

\end{document}
