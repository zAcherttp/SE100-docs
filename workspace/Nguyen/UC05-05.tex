% !TEX root = ../../main.tex
\documentclass[../../main.tex]{subfiles}

\begin{document}

\begin{table}[h]
\centering
\caption{Đặc tả use case}
\begin{tabular}{|p{4cm}|p{10cm}|}
\hline
\textbf{Mã UC} & UC05-05 \\
\hline
\textbf{Tên UC} & Xử lý sai lệch \\
\hline
\textbf{Mô tả tóm tắt} & So sánh thực tế vs số liệu PO và xử lý chênh lệch (thiếu/thừa). \\
\hline
\textbf{Tác nhân tham gia} & Chuyên viên nhận hàng, Quản lý kho \\
\hline
\textbf{Luồng sự kiện chính} & 
\begin{enumerate}
    \vspace{-18pt}
    \item Sau khi quét, hệ thống so sánh số lượng thực tế với PO.
    \item Nếu chênh lệch, nhân viên ghi nhận lý do và thêm ghi chú.
    \item Tạo báo cáo sai lệch hoặc yêu cầu điều chỉnh tồn kho.
\end{enumerate} \\
\hline
\textbf{Luồng sự kiện khác} & 
\begin{enumerate}
    \vspace{-18pt}
    \item[2a.] Chênh lệch lớn → yêu cầu xác minh của quản lý.
\end{enumerate} \\
\hline
\textbf{Yêu cầu đặc biệt} & Không có \\
\hline
\textbf{Trạng thái hệ thống ngay khi bắt đầu thực hiện} & Phiên nhập có dữ liệu so sánh với PO. \\
\hline
\textbf{Trạng thái hệ thống sau khi bắt đầu thực hiện} & Sai lệch được ghi nhận; nếu cần tạo yêu cầu điều chỉnh hay thông báo nhà cung cấp. \\
\hline
\textbf{Điểm mở rộng} & Tạo yêu cầu điều chỉnh (nếu cần) \\
\hline
\end{tabular}
\end{table}

\end{document}
