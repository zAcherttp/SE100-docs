\documentclass[../../main.tex]{subfiles}

\begin{document}
\begin{table}[h]
\centering
\begin{tabular}{|p{4cm}|p{10cm}|}
\hline
\textbf{Mã UC} & UC04-05 \\
\hline
\textbf{Tên UC} & Xem đơn đặt hàng \\
\hline
\textbf{Mô tả tóm tắt} & Người dùng có thể xem danh sách hoặc chi tiết các đơn đặt hàng đã tạo hoặc đã phê duyệt. \\
\hline
\textbf{Tác nhân tham gia} & Nhân viên thu mua, Quản lý kho, Kế toán \\
\hline
\textbf{Luồng sự kiện chính} &
\begin{enumerate}
    \vspace{-18pt}
    \item Mở chức năng “Xem đơn đặt hàng”.
    \item Hệ thống hiển thị danh sách đơn theo trạng thái.
    \item Người dùng có thể chọn để xem chi tiết từng đơn.
\end{enumerate} \\
\hline
\textbf{Luồng sự kiện khác} &
\begin{enumerate}
    \vspace{-18pt}
    \item[4a.] Nếu không có dữ liệu → hiển thị thông báo “Không có đơn hàng”. 
\end{enumerate} \\
\hline
\textbf{Yêu cầu đặc biệt} & Giao diện phải cho phép lọc theo trạng thái và nhà cung cấp. \\
\hline
\textbf{Trạng thái hệ thống ngay khi bắt đầu thực hiện} & Có dữ liệu đơn hàng trong hệ thống. \\
\hline
\textbf{Trạng thái hệ thống sau khi bắt đầu thực hiện} & Danh sách đơn hàng hiển thị chính xác. \\
\hline
\textbf{Điểm mở rộng} & Bao gồm UC04-06 Tính điểm đặt hàng lại (Hệ thống). \\
\hline
\end{tabular}

\caption{Bảng UC04-05}
\end{table}


\end{document}
