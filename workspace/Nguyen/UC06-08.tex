\documentclass[../../main.tex]{subfiles}

\begin{document}

\begin{table}[h]
\centering
\caption{Đặc tả use case}
\begin{tabular}{|p{4cm}|p{10cm}|}
\hline
\textbf{Mã UC} & UC06-08 \\
\hline
\textbf{Tên UC} & Xác thực phiên đóng gói \\
\hline
\textbf{Mô tả tóm tắt} & Quản lý kho kiểm tra và xác thực các phiên đóng gói để đảm bảo tính chính xác trước khi giao hàng. \\
\hline
\textbf{Tác nhân tham gia} & Quản lý kho \\
\hline
\textbf{Luồng sự kiện chính} &
\begin{enumerate}
    \vspace{-18pt}
    \item Quản lý mở danh sách phiên đóng gói.
    \item Kiểm tra số lượng, loại hàng và thông tin đóng gói.
    \item Xác thực hoặc từ chối phiên nếu có sai lệch.
\end{enumerate} \\
\hline
\textbf{Luồng sự kiện khác} &
\begin{enumerate}
    \vspace{-18pt}
    \item[4a.] Nếu từ chối → trả lại cho nhân viên để chỉnh sửa. 
\end{enumerate} \\
\hline
\textbf{Yêu cầu đặc biệt} & Chỉ người có quyền quản lý mới được xác thực. \\
\hline
\textbf{Trạng thái hệ thống ngay khi bắt đầu thực hiện} & Phiên đóng gói hoàn thành. \\
\hline
\textbf{Trạng thái hệ thống sau khi bắt đầu thực hiện} & Phiên đóng gói được xác thực và sẵn sàng giao hàng. \\
\hline
\textbf{Điểm mở rộng} & Bao gồm UC06-09 Tính tuyến đường lấy hàng tối ưu. \\
\hline
\end{tabular}
\end{table}

\end{document}
