\documentclass[../../main.tex]{subfiles}

\begin{document}

\begin{table}[h]
\centering
\begin{tabular}{|p{4cm}|p{10cm}|}
\hline
\textbf{Mã UC} & UC06-02 \\
\hline
\textbf{Tên UC} & Tạo phiên lấy hàng \\
\hline
\textbf{Mô tả tóm tắt} & Sau khi đơn hàng xuất được tạo, nhân viên kho khởi tạo phiên lấy hàng để chuẩn bị hàng hóa trong kho. \\
\hline
\textbf{Tác nhân tham gia} & Nhân viên kho \\
\hline
\textbf{Luồng sự kiện chính} & 
\begin{enumerate}
    \vspace{-18pt}
    \item Chọn đơn hàng xuất để tạo phiên lấy hàng.
    \item Xác định khu vực và người phụ trách lấy hàng.
    \item Hệ thống tạo phiên lấy hàng mới, gắn với đơn hàng tương ứng.
\end{enumerate} \\
\hline
\textbf{Luồng sự kiện khác} & 
\begin{enumerate}
    \vspace{-18pt}
    \item[4a.] Nếu không tìm thấy đơn hàng phù hợp → hiển thị cảnh báo.
\end{enumerate} \\
\hline
\textbf{Yêu cầu đặc biệt} & Phải có đơn hàng xuất đã được phê duyệt. \\
\hline
\textbf{Trạng thái hệ thống ngay khi bắt đầu thực hiện} & Đơn hàng xuất hợp lệ có trong hệ thống. \\
\hline
\textbf{Trạng thái hệ thống sau khi bắt đầu thực hiện} & Phiên lấy hàng được tạo và gắn với nhân viên phụ trách. \\
\hline
\textbf{Điểm mở rộng} & Bao gồm UC06-03 Thực thi lấy hàng. \\
\hline
\end{tabular}

\caption{Bảng UC06-02}
\end{table}

\end{document}
