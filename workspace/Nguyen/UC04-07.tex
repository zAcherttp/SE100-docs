\documentclass[../../main.tex]{subfiles}

\begin{document}

\begin{table}[h]
\centering
\begin{tabular}{|p{4cm}|p{10cm}|}
\hline
\textbf{Mã UC} & UC04-07 \\
\hline
\textbf{Tên UC} & Tạo cảnh báo tồn kho thấp (Hệ thống) \\
\hline
\textbf{Mô tả tóm tắt} & Khi số lượng hàng tồn thấp hơn ROP, hệ thống tự động tạo cảnh báo để nhắc người quản lý thực hiện mua hàng. \\
\hline
\textbf{Tác nhân tham gia} & (Hệ thống) \\
\hline
\textbf{Luồng sự kiện chính} &
\begin{enumerate}
    \vspace{-18pt}
    \item Hệ thống so sánh tồn kho thực tế với điểm ROP.
    \item Nếu tồn kho < ROP → tạo cảnh báo tồn kho thấp.
    \item Gửi thông báo đến nhân viên thu mua.
\end{enumerate} \\
\hline
\textbf{Luồng sự kiện khác} &
\begin{enumerate}
    \vspace{-18pt}
    \item[4a.] Nếu không cấu hình người nhận cảnh báo → lưu thông báo hệ thống. 
\end{enumerate} \\
\hline
\textbf{Yêu cầu đặc biệt} & Cần hệ thống gửi thông báo nội bộ hoạt động. \\
\hline
\textbf{Trạng thái hệ thống ngay khi bắt đầu thực hiện} & Có dữ liệu ROP hợp lệ. \\
\hline
\textbf{Trạng thái hệ thống sau khi bắt đầu thực hiện} & Cảnh báo tồn kho được tạo và hiển thị. \\
\hline
\textbf{Điểm mở rộng} & Bao gồm UC04-08 Tự động tính ROP. \\
\hline
\end{tabular}

\caption{Bảng UC04-07}
\end{table}

\end{document}
