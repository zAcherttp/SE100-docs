\documentclass[../../main.tex]{subfiles}

\begin{document}
\begin{table}[h]
\centering
\caption{Đặc tả use case}
\begin{tabular}{|p{4cm}|p{10cm}|}
\hline
\textbf{Mã UC} & UC04-03 \\
\hline
\textbf{Tên UC} & Gửi đơn đặt hàng \\
\hline
\textbf{Mô tả tóm tắt} & Nhân viên thu mua gửi đơn đặt hàng đã hoàn thiện lên quản lý hoặc kế toán để phê duyệt. \\
\hline
\textbf{Tác nhân tham gia} & Nhân viên thu mua, Kế toán \\
\hline
\textbf{Luồng sự kiện chính} &
\begin{enumerate}
    \vspace{-18pt}
    \item Chọn đơn hàng cần gửi.
    \item Nhấn “Gửi duyệt”.
    \item Hệ thống gửi thông báo đến người phê duyệt và cập nhật trạng thái đơn hàng thành “Chờ duyệt”.
\end{enumerate} \\
\hline
\textbf{Luồng sự kiện khác} &
\begin{enumerate}
    \vspace{-18pt}
    \item[4a.] Nếu thông tin đơn hàng chưa đầy đủ → không cho phép gửi.
\end{enumerate} \\
\hline
\textbf{Yêu cầu đặc biệt} & Cần cấu hình quyền gửi phê duyệt. \\
\hline
\textbf{Trạng thái hệ thống ngay khi bắt đầu thực hiện} & Đơn hàng ở trạng thái “Nháp”. \\
\hline
\textbf{Trạng thái hệ thống sau khi bắt đầu thực hiện} & Đơn hàng ở trạng thái “Chờ duyệt”. \\
\hline
\textbf{Điểm mở rộng} & Bao gồm UC04-04 Hủy đơn đặt hàng. \\
\hline
\end{tabular}
\end{table}

\end{document}
