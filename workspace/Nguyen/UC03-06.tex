\documentclass[../../main.tex]{subfiles}

\begin{document}
\begin{table}[h]
\centering
\caption{Bảng UC03-06}
\begin{tabular}{|p{4cm}|p{10cm}|}
\hline
\textbf{Mã UC} & UC03-06 \\
\hline
\textbf{Tên UC} & Trực quan hóa sơ đồ 2D \\
\hline
\textbf{Mô tả tóm tắt} & Hệ thống hiển thị sơ đồ kho dưới dạng bản đồ 2D giúp người dùng quan sát vị trí và cấu trúc kho dễ dàng. \\
\hline
\textbf{Tác nhân tham gia} & Quản lý kho, Nhân viên kho \\
\hline
\textbf{Luồng sự kiện chính} &
\begin{enumerate}
    \vspace{-18pt}
    \item Người dùng chọn chức năng “Hiển thị sơ đồ 2D”.
    \item Hệ thống tải dữ liệu bố cục kho.
    \item Vẽ sơ đồ kho lên giao diện theo tỷ lệ thực tế.
\end{enumerate} \\
\hline
\textbf{Luồng sự kiện khác} &
\begin{enumerate}
    \vspace{-18pt}
    \item[4a.] Nếu dữ liệu bố cục bị lỗi → hiển thị cảnh báo.
\end{enumerate} \\
\hline
\textbf{Yêu cầu đặc biệt} & Cần trình duyệt hỗ trợ đồ họa HTML5 Canvas hoặc SVG. \\
\hline
\textbf{Trạng thái hệ thống ngay khi bắt đầu thực hiện} & Dữ liệu bố cục kho đã tồn tại. \\
\hline
\textbf{Trạng thái hệ thống sau khi bắt đầu thực hiện} & Sơ đồ 2D được hiển thị trên giao diện người dùng. \\
\hline
\textbf{Điểm mở rộng} & Không có. \\
\hline
\end{tabular}
\end{table}

\end{document}
