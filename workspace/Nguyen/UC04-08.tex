\documentclass[../../main.tex]{subfiles}

\begin{document}

\begin{table}[h]
\centering
\begin{tabular}{|p{4cm}|p{10cm}|}
\hline
\textbf{Mã UC} & UC04-08 \\
\hline
\textbf{Tên UC} & Tự động tính ROP (Hệ thống) \\
\hline
\textbf{Mô tả tóm tắt} & Hệ thống tự động tính lại điểm đặt hàng (ROP) dựa trên lịch sử tiêu thụ hàng hóa theo thời gian thực. \\
\hline
\textbf{Tác nhân tham gia} & (Hệ thống) \\
\hline
\textbf{Luồng sự kiện chính} &
\begin{enumerate}
    \vspace{-18pt}
    \item Hệ thống lấy dữ liệu lịch sử tiêu thụ.
    \item Áp dụng công thức tính ROP dựa trên tốc độ tiêu thụ và thời gian giao hàng.
    \item Cập nhật giá trị ROP mới cho từng mặt hàng.
\end{enumerate} \\
\hline
\textbf{Luồng sự kiện khác} &
\begin{enumerate}
    \vspace{-18pt}
    \item[4a.] Nếu thiếu dữ liệu lịch sử → bỏ qua sản phẩm đó. 
\end{enumerate} \\
\hline
\textbf{Yêu cầu đặc biệt} & Hệ thống cần có module thống kê và dữ liệu thời gian thực. \\
\hline
\textbf{Trạng thái hệ thống ngay khi bắt đầu thực hiện} & Có dữ liệu tồn kho và lịch sử tiêu thụ. \\
\hline
\textbf{Trạng thái hệ thống sau khi bắt đầu thực hiện} & ROP của sản phẩm được cập nhật tự động. \\
\hline
\textbf{Điểm mở rộng} & Không có. \\
\hline
\end{tabular}

\caption{Bảng UC04-08}
\end{table}

\end{document}
