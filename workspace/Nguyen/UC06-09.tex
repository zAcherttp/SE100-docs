\documentclass[../../main.tex]{subfiles}

\begin{document}

\begin{table}[h]
\centering
\caption{Đặc tả use case}
\begin{tabular}{|p{4cm}|p{10cm}|}
\hline
\textbf{Mã UC} & UC06-09 \\
\hline
\textbf{Tên UC} & Tính tuyến đường lấy hàng tối ưu (Hệ thống) \\
\hline
\textbf{Mô tả tóm tắt} & Hệ thống tự động phân tích vị trí hàng trong kho để tính toán tuyến đường lấy hàng ngắn nhất và hiệu quả nhất. \\
\hline
\textbf{Tác nhân tham gia} & (Hệ thống) \\
\hline
\textbf{Luồng sự kiện chính} &
\begin{enumerate}
    \vspace{-18pt}
    \item Khi phiên lấy hàng được tạo, hệ thống truy xuất dữ liệu vị trí hàng.
    \item Tính toán lộ trình lấy hàng tối ưu dựa trên khoảng cách, khu vực và trọng lượng hàng.
    \item Hiển thị tuyến đường đề xuất cho nhân viên lấy hàng. 
\end{enumerate} \\
\hline
\textbf{Luồng sự kiện khác} &
\begin{enumerate}
    \vspace{-18pt}
    \item[4a.] Nếu thiếu dữ liệu vị trí → hệ thống hiển thị cảnh báo “Không đủ thông tin vị trí”. 
\end{enumerate} \\
\hline
\textbf{Yêu cầu đặc biệt} & Yêu cầu dữ liệu bản đồ kho chính xác. \\
\hline
\textbf{Trạng thái hệ thống ngay khi bắt đầu thực hiện} & Có thông tin vị trí sản phẩm trong kho. \\
\hline
\textbf{Trạng thái hệ thống sau khi bắt đầu thực hiện} & Hiển thị tuyến đường lấy hàng tối ưu trên giao diện. \\
\hline
\textbf{Điểm mở rộng} & Không có. \\
\hline
\end{tabular}
\end{table}

\end{document}
