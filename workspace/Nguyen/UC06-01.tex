\documentclass[../../main.tex]{subfiles}

\begin{document}

\begin{table}[h]
\centering
\caption{Bảng UC06-01}
\begin{tabular}{|p{4cm}|p{10cm}|}
\hline
\textbf{Mã UC} & UC06-01 \\
\hline
\textbf{Tên UC} & Tạo đơn hàng xuất \\
\hline
\textbf{Mô tả tóm tắt} & Chuyên viên giao hàng hoặc nhân viên kho tạo một đơn hàng xuất mới để chuẩn bị hàng hóa giao cho khách hàng hoặc chi nhánh khác. \\
\hline
\textbf{Tác nhân tham gia} & Chuyên viên giao hàng, Nhân viên kho \\
\hline
\textbf{Luồng sự kiện chính} & 
\begin{enumerate}
    \vspace{-18pt}
    \item Người dùng chọn chức năng “Tạo đơn hàng xuất”.
    \item Nhập thông tin khách hàng, mặt hàng và số lượng cần xuất.
    \item Hệ thống kiểm tra tồn kho.
    \item Hệ thống tạo đơn hàng xuất và chuyển sang trạng thái “Chờ xử lý”.
\end{enumerate} \\
\hline
\textbf{Luồng sự kiện khác} & 
\begin{enumerate}
    \vspace{-18pt}
    \item[4a.] Nếu tồn kho không đủ → hệ thống hiển thị thông báo và không tạo đơn hàng.
\end{enumerate} \\
\hline
\textbf{Yêu cầu đặc biệt} & Người dùng cần có quyền tạo đơn xuất. \\
\hline
\textbf{Trạng thái hệ thống ngay khi bắt đầu thực hiện} & Có hàng tồn kho sẵn sàng để xuất. \\
\hline
\textbf{Trạng thái hệ thống sau khi bắt đầu thực hiện} & Đơn hàng xuất được tạo và chờ xử lý tiếp theo. \\
\hline
\textbf{Điểm mở rộng} & Bao gồm UC06-02 Tạo phiên lấy hàng. \\
\hline
\end{tabular}
\end{table}

\end{document}
