\documentclass[../UC.tex]{subfiles}

\begin{document}

\begin{table}[h]
\centering
\begin{tabular}{|p{4cm}|p{10cm}|}
\hline
\textbf{Mã UC} & UC07-10 \\
\hline
\textbf{Tên UC} & Kiểm tra Hạn sử dụng (Hệ thống) \\
\hline
\textbf{Mô tả tóm tắt} & Hệ thống tự động quét và cảnh báo các lô hàng sắp hết hạn. \\
\hline
\textbf{Tác nhân tham gia} & System (Hệ thống - Tác vụ tự động) \\
\hline
\textbf{Luồng sự kiện chính} & 
\begin{enumerate}
    \vspace{-18pt}
    \item Hệ thống truy vấn tất cả các lô hàng (Batches) có expires_at và số lượng > 0.
    \item Hệ thống so sánh expires_at với ngày hiện tại và các ngưỡng (7, 30, 60 ngày).
    \item Với mỗi lô hàng vi phạm, hệ thống tạo một thông báo (UC11-05).
    \item Gửi thông báo đến Quản lý Kho.
\end{enumerate} \\
\hline
\textbf{Luồng sự kiện khác} & Không có \\
\hline
\textbf{Yêu cầu đặc biệt} & Tác vụ phải chạy ngầm, không ảnh hưởng hiệu suất ban ngày. \\
\hline
\textbf{Trạng thái hệ thống ngay khi bắt đầu thực hiện} & Tác vụ được lập lịch (cron job). \\
\hline
\textbf{Trạng thái hệ thống sau khi bắt đầu thực hiện} & Các thông báo cảnh báo hết hạn được tạo ra. \\
\hline
\textbf{Điểm mở rộng} & Không có \\
\hline
\end{tabular}

\caption{Bảng UC07-10}
\end{table}

\end{document}

