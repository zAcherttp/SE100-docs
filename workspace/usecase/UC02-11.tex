\documentclass[../UC.tex]{subfiles}

\begin{document}

\begin{table}[h]
\centering
\begin{tabular}{|p{4cm}|p{10cm}|}
\hline
\textbf{Mã UC} & UC02-11 \\
\hline
\textbf{Tên UC} & Create SKU (Tạo SKU) \\
\hline
\textbf{Mô tả tóm tắt} & Tạo một biến thể sản phẩm (SKU) cụ thể, có thể bán được, dựa trên một sản phẩm chủ. \\
\hline
\textbf{Tác nhân tham gia} & Quản lý kho / Quản trị viên hàng tồn kho \\
\hline
\textbf{Luồng sự kiện chính} & 
\begin{enumerate}
    \vspace{-18pt}
    \item Tác nhân điều hướng đến trang chi tiết của một Sản phẩm (Product).
    \item Tác nhân chọn "Thêm SKU / Biến thể".
    \item Hệ thống hiển thị biểu mẫu tạo SKU.
    \item Tác nhân «include» Generate/Assign SKU Code (tạo/gán mã SKU duy nhất).
    \item Tác nhân «include» Set Cost/Selling Price (đặt giá vốn/giá bán).
    \item Tác nhân «include» Define Dimensions (định nghĩa kích thước: trọng lượng, thể tích).
    \item Tác nhân «include» Set Packing Constraints (đặt ràng buộc đóng gói: nhiệt độ, giới hạn xếp chồng).
    \item Tác nhân nhấn "Lưu".
\end{enumerate} \\
\hline
\textbf{Luồng sự kiện khác} & 
\begin{enumerate}
    \vspace{-18pt}
    \item[4a.] Mã SKU bị trùng -> Hệ thống báo lỗi.
\end{enumerate} \\
\hline
\textbf{Yêu cầu đặc biệt} & SKU là đơn vị cơ sở để theo dõi tồn kho. \\
\hline
\textbf{Trạng thái hệ thống ngay khi bắt đầu thực hiện} & Tác nhân đã đăng nhập và có quyền quản lý sản phẩm. Sản phẩm chủ đã tồn tại. \\
\hline
\textbf{Trạng thái hệ thống sau khi bắt đầu thực hiện} & Một biến thể SKU mới được tạo và liên kết với sản phẩm chủ. \\
\hline
\textbf{Điểm mở rộng} & Không có \\
\hline
\end{tabular}

\caption{Bảng UC02-11}
\end{table}

\end{document}