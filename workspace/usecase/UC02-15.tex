\documentclass[../UC.tex]{subfiles}

\begin{document}

\begin{table}[h]
\centering
\begin{tabular}{|p{4cm}|p{10cm}|}
\hline
\textbf{Mã UC} & UC02-15 \\
\hline
\textbf{Tên UC} & Delete SKU (Xóa SKU) \\
\hline
\textbf{Mô tả tóm tắt} & Cho phép quản trị viên xóa mềm (soft delete) một SKU. \\
\hline
\textbf{Tác nhân tham gia} & Quản lý kho / Quản trị viên hàng tồn kho \\
\hline
\textbf{Luồng sự kiện chính} & 
\begin{enumerate}
    \vspace{-18pt}
    \item Tác nhân chọn một SKU và nhấn "Xóa".
    \item Hệ thống hiển thị xác nhận.
    \item Tác nhân xác nhận.
    \item Hệ thống thực hiện Soft Delete (cập nhật is_deleted = true).
\end{enumerate} \\
\hline
\textbf{Luồng sự kiện khác} & 
\begin{enumerate}
    \vspace{-18pt}
    \item[2a.] SKU đang có hàng tồn kho (inventory batches/serials) -> Hệ thống chặn xóa và thông báo "Không thể xóa SKU đang có tồn kho".
\end{enumerate} \\
\hline
\textbf{Yêu cầu đặc biệt} & Phải là xóa mềm. \\
\hline
\textbf{Trạng thái hệ thống ngay khi bắt đầu thực hiện} & Tác nhân đã đăng nhập. SKU đã tồn tại. \\
\hline
\textbf{Trạng thái hệ thống sau khi bắt đầu thực hiện} & SKU bị đánh dấu là đã xóa. \\
\hline
\textbf{Điểm mở rộng} & Không có \\
\hline
\end{tabular}

\caption{Bảng UC02-15}
\end{table}

\end{document}