\documentclass[../UC.tex]{subfiles}

\begin{document}

\begin{table}[h]
\centering
\begin{tabular}{|p{4cm}|p{10cm}|}
\hline
\textbf{Mã UC} & UC04-01 \\
\hline
\textbf{Tên UC} & Tạo đơn đặt hàng \\
\hline
\textbf{Mô tả tóm tắt} & Nhân viên thu mua tạo đơn đặt hàng mới với thông tin nhà cung cấp, sản phẩm và số lượng cần mua. \\
\hline
\textbf{Tác nhân tham gia} & Nhân viên thu mua, Kế toán \\
\hline
\textbf{Luồng sự kiện chính} &
\begin{enumerate}
    \vspace{-18pt}
    \item Chọn chức năng “Tạo đơn đặt hàng”.
    \item Nhập thông tin nhà cung cấp, sản phẩm, số lượng, giá.
    \item Hệ thống lưu đơn hàng ở trạng thái “Nháp”.
\end{enumerate} \\
\hline
\textbf{Luồng sự kiện khác} &
\begin{enumerate}
    \vspace{-18pt}
    \item[4a.] Nếu thiếu thông tin bắt buộc → hiển thị cảnh báo.
\end{enumerate} \\
\hline
\textbf{Yêu cầu đặc biệt} & Người tạo phải có quyền nhập dữ liệu đơn mua hàng. \\
\hline
\textbf{Trạng thái hệ thống ngay khi bắt đầu thực hiện} & Hệ thống có dữ liệu sản phẩm và nhà cung cấp. \\
\hline
\textbf{Trạng thái hệ thống sau khi bắt đầu thực hiện} & Đơn đặt hàng mới được tạo và lưu trong hệ thống. \\
\hline
\textbf{Điểm mở rộng} & Bao gồm UC04-02 Sửa đơn đặt hàng. \\
\hline
\end{tabular}

\caption{Bảng UC04-01}
\end{table}

\end{document}
