\documentclass[../UC.tex]{subfiles}

\begin{document}

\begin{table}[htbp]
\centering
\begin{tabular}{|p{4cm}|p{10cm}|}
\hline
\textbf{Mã UC} & UC07-06 \\
\hline
\textbf{Tên UC} & Tạo yêu cầu điều chỉnh \\
\hline
\textbf{Mô tả tóm tắt} & Nhân viên hoặc quản lý tạo yêu cầu điều chỉnh hàng tồn khi phát hiện sai lệch giữa dữ liệu hệ thống và kiểm kê thực tế. \\
\hline
\textbf{Tác nhân tham gia} & Quản lý kho, Nhân viên kho \\
\hline
\textbf{Luồng sự kiện chính} &
\begin{enumerate}
    \vspace{-18pt}
    \item Chọn chức năng “Tạo yêu cầu điều chỉnh”.
    \item Nhập thông tin sản phẩm, số lượng chênh lệch và lý do.
    \item Hệ thống tạo yêu cầu điều chỉnh và lưu vào danh sách chờ phê duyệt.
\end{enumerate} \\
\hline
\textbf{Luồng sự kiện khác} &
\begin{enumerate}
    \vspace{-18pt}
    \item[4a.] Nếu thông tin không hợp lệ → hiển thị cảnh báo. 
\end{enumerate} \\
\hline
\textbf{Yêu cầu đặc biệt} & Chỉ được tạo điều chỉnh cho sản phẩm đã xác thực kiểm kê. \\
\hline
\textbf{Trạng thái hệ thống ngay khi bắt đầu thực hiện} & Đã phát hiện chênh lệch trong kiểm kê. \\
\hline
\textbf{Trạng thái hệ thống sau khi bắt đầu thực hiện} & Yêu cầu điều chỉnh được tạo thành công. \\
\hline
\textbf{Điểm mở rộng} & Bao gồm UC07-07 Gửi yêu cầu điều chỉnh để phê duyệt. \\
\hline
\end{tabular}

\caption{Bảng UC07-06}
\end{table}

\end{document}
