\documentclass[../UC.tex]{subfiles}

\begin{document}

\begin{table}[h]
\centering
\begin{tabular}{|p{4cm}|p{10cm}|}
\hline
\textbf{Mã UC} & UC03-03 \\
\hline
\textbf{Tên UC} & Edit Storage Zone (Chỉnh sửa Khu vực Lưu trữ) \\
\hline
\textbf{Mô tả tóm tắt} & Cho phép quản lý thay đổi vị trí, kích thước, hoặc các thuộc tính của một khu vực lưu trữ đã tồn tại. \\
\hline
\textbf{Tác nhân tham gia} & Quản lý kho / Quản trị viên hàng tồn kho \\
\hline
\textbf{Luồng sự kiện chính} & 
\begin{enumerate}
    \vspace{-18pt}
    \item Tác nhân nhấp chuột vào một khu vực trên sơ đồ 2D.
    \item Tác nhân chọn "Chỉnh sửa".
    \item Tác nhân thực hiện thay đổi:
    \item[3a.] Kéo thả để thay đổi vị trí (Move Zone Position).
    \item[3b.] Kéo các cạnh để thay đổi kích thước (Resize Zone).
    \item[3c.] Xoay khu vực (Rotate Zone).
    \item[3d.] Cập nhật thuộc tính trong biểu mẫu (Update Attributes, ví dụ: nhiệt độ).
    \item Tác nhân nhấn "Lưu".
    \item Hệ thống cập nhật bản ghi khu vực (tọa độ, JSONB attributes).
\end{enumerate} \\
\hline
\textbf{Luồng sự kiện khác} & 
\begin{enumerate}
    \vspace{-18pt}
    \item[3a.] Khu vực đang chứa hàng tồn kho -> Hệ thống có thể chặn thay đổi các thuộc tính quan trọng (như nhiệt độ) hoặc yêu cầu xác nhận.
\end{enumerate} \\
\hline
\textbf{Yêu cầu đặc biệt} & Không có \\
\hline
\textbf{Trạng thái hệ thống ngay khi bắt đầu thực hiện} & Tác nhân đã đăng nhập và có quyền "CONFIGURE_WAREHOUSE". Khu vực đã tồn tại. \\
\hline
\textbf{Trạng thái hệ thống sau khi bắt đầu thực hiện} & Thông tin/vị trí của khu vực lưu trữ được cập nhật. \\
\hline
\textbf{Điểm mở rộng} & Không có \\
\hline
\end{tabular}

\caption{Bảng UC03-03}
\end{table}

\end{document}