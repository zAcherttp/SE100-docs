\documentclass[../UC.tex]{subfiles}

\begin{document}

\begin{table}[h]
\centering
\begin{tabular}{|p{4cm}|p{10cm}|}
\hline
\textbf{Mã UC} & UC03-04 \\
\hline
\textbf{Tên UC} & Xóa khu vực lưu trữ \\
\hline
\textbf{Mô tả tóm tắt} & Quản trị viên hoặc quản lý kho có thể xóa khu vực lưu trữ không còn được sử dụng trong sơ đồ kho. \\
\hline
\textbf{Tác nhân tham gia} & Quản trị viên hệ thống, Quản lý kho \\
\hline
\textbf{Luồng sự kiện chính} &
\begin{enumerate}
    \vspace{-18pt}
    \item Người dùng chọn khu vực cần xóa trong danh sách khu vực lưu trữ.
    \item Xác nhận thao tác xóa.
    \item Hệ thống kiểm tra xem khu vực có đang chứa hàng hay không.
    \item Nếu hợp lệ, khu vực được xóa khỏi sơ đồ kho và cơ sở dữ liệu.
\end{enumerate} \\
\hline
\textbf{Luồng sự kiện khác} &
\begin{enumerate}
    \vspace{-18pt}
    \item[4a.] Nếu khu vực đang chứa hàng → hiển thị cảnh báo và không cho phép xóa.
\end{enumerate} \\
\hline
\textbf{Yêu cầu đặc biệt} & Chỉ người có quyền quản lý kho mới được phép xóa khu vực. \\
\hline
\textbf{Trạng thái hệ thống ngay khi bắt đầu thực hiện} & Khu vực đã tồn tại trong hệ thống. \\
\hline
\textbf{Trạng thái hệ thống sau khi bắt đầu thực hiện} & Khu vực được xóa khỏi sơ đồ và dữ liệu kho. \\
\hline
\textbf{Điểm mở rộng} & Không có. \\
\hline
\end{tabular}

\caption{Bảng UC03-04}
\end{table}

\end{document}
