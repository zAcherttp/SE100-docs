\documentclass[../UC.tex]{subfiles}

\begin{document}

\begin{table}[h]
\centering
\begin{tabular}{|p{4cm}|p{10cm}|}
\hline
\textbf{Mã UC} & UC04-04 \\
\hline
\textbf{Tên UC} & Hủy đơn đặt hàng \\
\hline
\textbf{Mô tả tóm tắt} & Người dùng hủy đơn đặt hàng chưa được phê duyệt hoặc khi phát hiện lỗi nhập liệu. \\
\hline
\textbf{Tác nhân tham gia} & Nhân viên thu mua, Quản lý kho \\
\hline
\textbf{Luồng sự kiện chính} &
\begin{enumerate}
    \vspace{-18pt}
    \item Chọn đơn hàng cần hủy.
    \item Xác nhận lý do hủy.
    \item Hệ thống cập nhật trạng thái đơn hàng thành “Đã hủy”.
\end{enumerate} \\
\hline
\textbf{Luồng sự kiện khác} &
\begin{enumerate}
    \vspace{-18pt}
    \item[4a.] Nếu đơn hàng đã được duyệt → hiển thị thông báo “Không thể hủy”.
\end{enumerate} \\
\hline
\textbf{Yêu cầu đặc biệt} & Cần lưu lại lịch sử hủy đơn hàng. \\
\hline
\textbf{Trạng thái hệ thống ngay khi bắt đầu thực hiện} & Đơn hàng đang ở trạng thái “Nháp” hoặc “Chờ duyệt”. \\
\hline
\textbf{Trạng thái hệ thống sau khi bắt đầu thực hiện} & Đơn hàng bị hủy và ghi nhận vào lịch sử. \\
\hline
\textbf{Điểm mở rộng} & Bao gồm UC04-05 Xem đơn đặt hàng. \\
\hline
\end{tabular}

\caption{Bảng UC04-04}
\end{table}

\end{document}
