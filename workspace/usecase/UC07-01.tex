\documentclass[../UC.tex]{subfiles}

\begin{document}

\begin{table}[htbp]
\centering
\begin{tabular}{|p{4cm}|p{10cm}|}
\hline
\textbf{Mã UC} & UC07-01 \\
\hline
\textbf{Tên UC} & Tạo phiên kiểm kê \\
\hline
\textbf{Mô tả tóm tắt} & Nhân viên kho hoặc quản lý kho khởi tạo một phiên kiểm kê mới để thực hiện đối chiếu giữa hàng tồn thực tế và dữ liệu hệ thống. \\
\hline
\textbf{Tác nhân tham gia} & Nhân viên kho, Quản lý kho \\
\hline
\textbf{Luồng sự kiện chính} &
\begin{enumerate}
    \vspace{-18pt}
    \item Chọn chức năng “Tạo phiên kiểm kê” tại trang phiên kiểm kê trong kho.
    \item Nhập thông tin: kiểm kê hằng ngày hoặc từng tuần.
    \item Điền các thông tin cần thiết cho phiên kiểm kê.
    \item Xác nhận tạo phiên kiểm kê.
\end{enumerate} \\
\hline
\textbf{Luồng sự kiện khác} &
\begin{enumerate}
    \vspace{-18pt}
    \item[4a.] Nếu thiếu thông tin cần thiết → hiển thị cảnh báo và yêu cầu bổ sung. 
\end{enumerate} \\
\hline
\textbf{Yêu cầu đặc biệt} & Người khởi tạo phải có quyền thực hiện kiểm kê. \\
\hline
\textbf{Trạng thái hệ thống ngay khi bắt đầu thực hiện} & Hệ thống đang hoạt động bình thường. \\
\hline
\textbf{Trạng thái hệ thống sau khi bắt đầu thực hiện} & Phiên kiểm kê được khởi tạo thành công. \\
\hline
\textbf{Điểm mở rộng} & Không. \\
\hline
\end{tabular}

\caption{Bảng UC07-01}
\end{table}

\end{document}
