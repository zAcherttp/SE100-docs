\documentclass[../UC.tex]{subfiles}

\begin{document}

\begin{table}[htbp]
\centering
\begin{tabular}{|p{4cm}|p{10cm}|}
\hline
\textbf{Mã UC} & UC07-15 \\
\hline
\textbf{Tên UC} & Xem thông tin chi tiết của yêu cầu điều chỉnh \\
\hline
\textbf{Mô tả tóm tắt} & Xem thông tin chi tiết của yêu cầu điều chỉnh. \\
\hline
\textbf{Tác nhân tham gia} & Nhân viên kho, Quản lý kho \\
\hline
\textbf{Luồng sự kiện chính} &
\begin{enumerate}
    \vspace{-18pt}
    \item Truy cập vào trang yêu cầu điều chỉnh trong kho.
    \item Chọn mục yêu cầu cần xem (số lượng và vị trí).
    \item Chọn xem chi tiết tại yêu cầu cần kiểm tra.
    \item Xem đầy đủ thông tin chi tiết của yêu cầu điều chỉnh.
\end{enumerate} \\
\hline
\textbf{Luồng sự kiện khác} &
\begin{enumerate}
    \vspace{-18pt}
    \item[4a.] Không. 
\end{enumerate} \\
\hline
\textbf{Yêu cầu đặc biệt} & Người xem phải có quyền truy cập thông tin yêu cầu điều chỉnh. \\
\hline
\textbf{Trạng thái hệ thống ngay khi bắt đầu thực hiện} & Hệ thống phải có dữ liệu yêu cầu điều chỉnh. \\
\hline
\textbf{Trạng thái hệ thống sau khi bắt đầu thực hiện} & Thông tin chi tiết yêu cầu điều chỉnh được hiển thị thành công. \\
\hline
\textbf{Điểm mở rộng} & Không. \\
\hline
\end{tabular}

\caption{Bảng UC07-15}
\end{table}

\end{document}
