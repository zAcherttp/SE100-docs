\documentclass[../UC.tex]{subfiles}

\begin{document}

\begin{table}[h]
\centering
\begin{tabular}{|p{4cm}|p{10cm}|}
\hline
\textbf{Mã UC} & UC02-04 \\
\hline
\textbf{Tên UC} & Delete Category (Xóa Danh mục) \\
\hline
\textbf{Mô tả tóm tắt} & Cho phép quản trị viên xóa mềm (soft delete) một danh mục. \\
\hline
\textbf{Tác nhân tham gia} & Quản lý kho / Quản trị viên hàng tồn kho \\
\hline
\textbf{Luồng sự kiện chính} & 
\begin{enumerate}
    \vspace{-18pt}
    \item Tác nhân điều hướng đến "Dữ liệu chủ" -> "Danh mục".
    \item Tác nhân chọn một danh mục và nhấn "Xóa".
    \item Hệ thống hiển thị xác nhận.
    \item Tác nhân xác nhận.
    \item Hệ thống thực hiện Soft Delete (cập nhật is_deleted = true, deleted_at = NOW()).
\end{enumerate} \\
\hline
\textbf{Luồng sự kiện khác} & 
\begin{enumerate}
    \vspace{-18pt}
    \item[3a.] Tác nhân hủy bỏ -> Không có gì xảy ra.
    \item[3b.] Danh mục đang chứa sản phẩm -> Hệ thống cảnh báo "Danh mục chứa sản phẩm. Vui lòng di chuyển sản phẩm sang danh mục khác trước khi xóa." (Hoặc tự động gán sản phẩm về "Uncategorized").
\end{enumerate} \\
\hline
\textbf{Yêu cầu đặc biệt} & Phải là xóa mềm để bảo toàn lịch sử dữ liệu. \\
\hline
\textbf{Trạng thái hệ thống ngay khi bắt đầu thực hiện} & Tác nhân đã đăng nhập và có quyền "MANAGE_CATEGORIES". \\
\hline
\textbf{Trạng thái hệ thống sau khi bắt đầu thực hiện} & Danh mục bị đánh dấu là đã xóa và không hiển thị trong các lựa chọn thông thường. \\
\hline
\textbf{Điểm mở rộng} & Không có \\
\hline
\end{tabular}

\caption{Bảng UC02-04}
\end{table}

\end{document}