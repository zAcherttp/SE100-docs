\documentclass[../UC.tex]{subfiles}

\begin{document}

\begin{table}[h]
\centering
\begin{tabular}{|p{4cm}|p{10cm}|}
\hline
\textbf{Mã UC} & UC02-12 \\
\hline
\textbf{Tên UC} & Manage Barcodes (Quản lý Mã vạch) \\
\hline
\textbf{Mô tả tóm tắt} & Cho phép gán nhiều mã vạch (UPC, EAN, nội bộ, QR) cho một SKU. \\
\hline
\textbf{Tác nhân tham gia} & Quản lý kho / Quản trị viên hàng tồn kho \\
\hline
\textbf{Luồng sự kiện chính} & 
\begin{enumerate}
    \vspace{-18pt}
    \item Tác nhân điều hướng đến trang chi tiết của một SKU.
    \item Tác nhân chọn "Quản lý Mã vạch".
    \item Tác nhân «include» Add Multiple Barcodes (thêm giá trị mã vạch mới).
    \item Tác nhân «include» Specify Barcode Type (chỉ định loại: UPC, EAN, QR, Internal).
    \item Tác nhân nhấn "Lưu".
    \item (Tùy chọn) Hệ thống «include» Generate QR Code (tự động tạo QR code nếu là loại 'Internal').
\end{enumerate} \\
\hline
\textbf{Luồng sự kiện khác} & 
\begin{enumerate}
    \vspace{-18pt}
    \item[3a.] Mã vạch đã tồn tại (được gán cho SKU khác) -> Hệ thống báo lỗi.
\end{enumerate} \\
\hline
\textbf{Yêu cầu đặc biệt} & Cần có bảng product_barcodes riêng để lưu quan hệ nhiều-một với SKU. \\
\hline
\textbf{Trạng thái hệ thống ngay khi bắt đầu thực hiện} & Tác nhân đã đăng nhập. SKU đã tồn tại. \\
\hline
\textbf{Trạng thái hệ thống sau khi bắt đầu thực hiện} & SKU được liên kết với một hoặc nhiều mã vạch. \\
\hline
\textbf{Điểm mở rộng} & Không có \\
\hline
\end{tabular}

\caption{Bảng UC02-12}
\end{table}

\end{document}