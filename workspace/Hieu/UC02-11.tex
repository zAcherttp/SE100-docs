\documentclass[../main.tex]{subfiles}

\begin{document}

\begin{table}[h]
\centering
\caption{Bảng UC02-11}
\begin{tabular}{|p{4cm}|p{10cm}|}
\hline
\textbf{Mã UC} & UC02-11 \\
\hline
\textbf{Tên UC} & Delete Product (Xóa Sản phẩm) \\
\hline
\textbf{Mô tả tóm tắt} & Cho phép quản trị viên xóa mềm (soft delete) một sản phẩm chủ. \\
\hline
\textbf{Tác nhân tham gia} & Quản lý kho / Quản trị viên hàng tồn kho \\
\hline
\textbf{Luồng sự kiện chính} & 
\begin{enumerate}
    \vspace{-18pt}
    \item Tác nhân chọn một sản phẩm và nhấn "Xóa".
    \item Hệ thống hiển thị xác nhận.
    \item Tác nhân xác nhận.
    \item Hệ thống thực hiện Soft Delete (cập nhật is_deleted = true).
\end{enumerate} \\
\hline
\textbf{Luồng sự kiện khác} & 
\begin{enumerate}
    \vspace{-18pt}
    \item[2a.] Sản phẩm đang có SKU liên kết hoặc hàng tồn kho -> Hệ thống chặn xóa và yêu cầu xóa SKU trước.
\end{enumerate} \\
\hline
\textbf{Yêu cầu đặc biệt} & Phải là xóa mềm. \\
\hline
\textbf{Trạng thái hệ thống ngay khi bắt đầu thực hiện} & Tác nhân đã đăng nhập và có quyền quản lý sản phẩm. \\
\hline
\textbf{Trạng thái hệ thống sau khi bắt đầu thực hiện} & Sản phẩm chủ bị đánh dấu là đã xóa. \\
\hline
\textbf{Điểm mở rộng} & Không có \\
\hline
\end{tabular}
\end{table}

\end{document}