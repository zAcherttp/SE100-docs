\documentclass[../main.tex]{subfiles}

\begin{document}

\begin{table}[h]
\centering
\caption{Bảng UC02-07}
\begin{tabular}{|p{4cm}|p{10cm}|}
\hline
\textbf{Mã UC} & UC02-07 \\
\hline
\textbf{Tên UC} & Delete Brand (Xóa Thương hiệu) \\
\hline
\textbf{Mô tả tóm tắt} & Cho phép quản trị viên xóa mềm (soft delete) một thương hiệu. \\
\hline
\textbf{Tác nhân tham gia} & Quản lý kho / Quản trị viên hàng tồn kho \\
\hline
\textbf{Luồng sự kiện chính} & 
\begin{enumerate}
    \vspace{-18pt}
    \item Tác nhân điều hướng đến "Dữ liệu chủ" -> "Thương hiệu".
    \item Tác nhân chọn một thương hiệu và nhấn "Xóa".
    \item Hệ thống hiển thị xác nhận.
    \item Tác nhân xác nhận.
    \item Hệ thống thực hiện Soft Delete (cập nhật is_deleted = true).
\end{enumerate} \\
\hline
\textbf{Luồng sự kiện khác} & 
\begin{enumerate}
    \vspace{-18pt}
    \item[3a.] Thương hiệu đang được gán cho sản phẩm -> Hệ thống cảnh báo "Không thể xóa thương hiệu đang được sử dụng".
\end{enumerate} \\
\hline
\textbf{Yêu cầu đặc biệt} & Phải là xóa mềm. \\
\hline
\textbf{Trạng thái hệ thống ngay khi bắt đầu thực hiện} & Tác nhân đã đăng nhập và có quyền quản lý dữ liệu chủ. \\
\hline
\textbf{Trạng thái hệ thống sau khi bắt đầu thực hiện} &Thương hiệu bị đánh dấu là đã xóa. \\
\hline
\textbf{Điểm mở rộng} & Không có \\
\hline
\end{tabular}
\end{table}

\end{document}