\documentclass[../main.tex]{subfiles}

\begin{document}

\begin{table}[h]
\centering
\begin{tabular}{|p{4cm}|p{10cm}|}
\hline
\textbf{Mã UC} & UC02-08 \\
\hline
\textbf{Tên UC} & Create Product (Tạo Sản phẩm) \\
\hline
\textbf{Mô tả tóm tắt} & Tạo một bản ghi sản phẩm chủ (master record), xác định các thuộc tính chung trước khi tạo các biến thể (SKU). \\
\hline
\textbf{Tác nhân tham gia} & Quản lý kho / Quản trị viên hàng tồn kho \\
\hline
\textbf{Luồng sự kiện chính} & 
\begin{enumerate}
    \vspace{-18pt}
    \item Tác nhân điều hướng đến "Dữ liệu chủ" -> "Sản phẩm" và chọn "Tạo mới".
    \item Hệ thống hiển thị biểu mẫu tạo sản phẩm.
    \item Tác nhân nhập Tên, Mô tả.
    \item Tác nhân «include» Assign Category (gán vào danh mục).
    \item Tác nhân «include» Assign Brand (gán vào thương hiệu).
    \item Tác nhân «include» Set Storage Requirements (đặt yêu cầu lưu trữ: khô, lạnh...).
    \item Tác nhân «include» Set Tracking Method (chọn SERIAL, BATCH, hoặc NONE).
    \item Tác nhân nhấn "Lưu".
    \item Hệ thống tạo bản ghi sản phẩm chủ.
\end{enumerate} \\
\hline
\textbf{Luồng sự kiện khác} & 
\begin{enumerate}
    \vspace{-18pt}
    \item[8a.] Thiếu thông tin bắt buộc -> Hệ thống báo lỗi.
\end{enumerate} \\
\hline
\textbf{Yêu cầu đặc biệt} & Không có \\
\hline
\textbf{Trạng thái hệ thống ngay khi bắt đầu thực hiện} & Tác nhân đã đăng nhập và có quyền quản lý sản phẩm. Danh mục và Thương hiệu đã tồn tại. \\
\hline
\textbf{Trạng thái hệ thống sau khi bắt đầu thực hiện} & Một bản ghi sản phẩm chủ được tạo. \\
\hline
\textbf{Điểm mở rộng} & Không có \\
\hline
\end{tabular}

\caption{Bảng UC02-08}
\end{table}

\end{document}