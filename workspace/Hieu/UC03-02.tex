\documentclass[../main.tex]{subfiles}

\begin{document}

\begin{table}[h]
\centering
\caption{Bảng UC03-02}
\begin{tabular}{|p{4cm}|p{10cm}|}
\hline
\textbf{Mã UC} & UC03-02 \\
\hline
\textbf{Tên UC} & Create Storage Zone (Tạo Khu vực Lưu trữ) \\
\hline
\textbf{Mô tả tóm tắt} & Cho phép quản lý thêm một khu vực lưu trữ (zone) vào sơ đồ kho, định nghĩa cấu trúc phân cấp và thuộc tính của nó. \\
\hline
\textbf{Tác nhân tham gia} & Quản lý kho / Quản trị viên hàng tồn kho \\
\hline
\textbf{Luồng sự kiện chính} & 
\begin{enumerate}
    \vspace{-18pt}
    \item Tác nhân đang ở giao diện Sơ đồ kho (từ UC03-01).
    \item Tác nhân kéo/thả một khu vực mới vào sơ đồ.
    \item Hệ thống hiển thị biểu mẫu cấu hình khu vực.
    \item Tác nhân «include» Select Storage Block Type (chọn mẫu phân cấp: Kệ khô, Tủ đông, Sàn...).
    \item Hệ thống tự động/Tác nhân «include» Set ltree Path (đặt đường dẫn ltree, ví dụ: 'KHO_A.KE_01.TANG_2').
    \item Tác nhân «include» Assign Zone Type (chọn loại khu vực: RECEIVING, STORAGE, PACKING, QUARANTINE...).
    \item Tác nhân «include» Set Zone Attributes (JSONB) (đặt thuộc tính: tọa độ, kích thước, nhiệt độ, tải trọng...).
    \item Tác nhân nhấn "Lưu".
\end{enumerate} \\
\hline
\textbf{Luồng sự kiện khác} & 
\begin{enumerate}
    \vspace{-18pt}
    \item[7a.] Tọa độ hoặc kích thước khu vực mới bị chồng chéo lên khu vực đã có -> Hệ thống báo lỗi.
\end{enumerate} \\
\hline
\textbf{Yêu cầu đặc biệt} & Sử dụng ltree cho phân cấp và JSONB cho các thuộc tính linh hoạt. \\
\hline
\textbf{Trạng thái hệ thống ngay khi bắt đầu thực hiện} & Tác nhân đã đăng nhập và có quyền "CONFIGURE_WAREHOUSE". Sơ đồ 2D đã tồn tại. \\
\hline
\textbf{Trạng thái hệ thống sau khi bắt đầu thực hiện} & Một khu vực lưu trữ mới được thêm vào sơ đồ kho. \\
\hline
\textbf{Điểm mở rộng} & Không có \\
\hline
\end{tabular}
\end{table}

\end{document}