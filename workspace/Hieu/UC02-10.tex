\documentclass[../main.tex]{subfiles}

\begin{document}

\begin{table}[h]
\centering
\caption{Đặc tả use case}
\begin{tabular}{|p{4cm}|p{10cm}|}
\hline
\textbf{Mã UC} & UC02-10 \\
\hline
\textbf{Tên UC} & Define Custom Product Fields (Định nghĩa Trường tùy chỉnh cho Sản phẩm) \\
\hline
\textbf{Mô tả tóm tắt} & Cho phép quản trị viên tạo các mẫu (template) để thêm các thuộc tính riêng cho từng loại sản phẩm (ví dụ: "Màu sắc", "Kích thước" cho "Quần áo"). \\
\hline
\textbf{Tác nhân tham gia} & Quản lý kho / Quản trị viên hàng tồn kho \\
\hline
\textbf{Luồng sự kiện chính} & 
\begin{enumerate}
    \vspace{-18pt}
    \item Tác nhân điều hướng đến "Cài đặt" -> "Mẫu Sản phẩm".
    \item Tác nhân «include» Create Product Type Template (Tạo mẫu loại sản phẩm, ví dụ: "Điện tử").
    \item Tác nhân thêm các trường vào mẫu.
    \item Với mỗi trường, tác nhân «include» Define Field Types (định nghĩa kiểu dữ liệu: text, number, date, dropdown).
    \item Tác nhân nhấn "Lưu".
    \item Định nghĩa mẫu được lưu (thường là trong CSDL, các giá trị sẽ được lưu vào JSONB của SKU).
\end{enumerate} \\
\hline
\textbf{Luồng sự kiện khác} & Không có \\
\hline
\textbf{Yêu cầu đặc biệt} & Không có \\
\hline
\textbf{Trạng thái hệ thống ngay khi bắt đầu thực hiện} & Tác nhân đã đăng nhập và có quyền quản lý sản phẩm. \\
\hline
\textbf{Trạng thái hệ thống sau khi bắt đầu thực hiện} & Một mẫu loại sản phẩm với các trường tùy chỉnh được định nghĩa. \\
\hline
\textbf{Điểm mở rộng} & Không có \\
\hline
\end{tabular}
\end{table}

\end{document}