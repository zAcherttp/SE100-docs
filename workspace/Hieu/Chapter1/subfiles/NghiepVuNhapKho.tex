\documentclass[../main.tex]{subfiles}
\begin{document}

\subsection{Nghiệp vụ nhập kho}

\begin{itemize}
    \item \textbf{Tiếp nhận chứng từ nhập kho:}
    \begin{itemize}
        \item Nhận phiếu nhập kho, hóa đơn mua hàng, hoặc các chứng từ liên quan từ bộ phận mua hàng hoặc nhà cung cấp.
        \item Kiểm tra tính hợp lệ và đầy đủ của các chứng từ.
    \end{itemize}
    
    \item \textbf{Kiểm tra hàng hóa thực tế:}
    \begin{itemize}
        \item Đối chiếu số lượng, chủng loại, quy cách, chất lượng hàng hóa thực tế với thông tin trên chứng từ.
        \item Phát hiện và lập bản xử lý các trường hợp sai lệch, hư hỏng (nếu có).
    \end{itemize}
    
    \item \textbf{Cập nhật thông tin nhập kho:}
    \begin{itemize}
        \item Ghi nhận thông tin hàng hóa vào sổ sách kho hoặc phần mềm quản lý kho.
        \item Cập nhật số lượng tồn kho.
        \item Gán mã vị trí lưu trữ cho hàng hóa.
    \end{itemize}
    
    \item \textbf{Sắp xếp và bố trí hàng hóa:}
    \begin{itemize}
        \item Di chuyển hàng hóa vào khu vực lưu trữ đã được xác định.
        \item Sắp xếp hàng hóa một cách khoa học, đảm bảo dễ dàng tìm kiếm, kiểm tra và xuất kho sau này.
        \item Tuân thủ các quy tắc về bảo quản và an toàn kho bãi.
    \end{itemize}
    
    \item \textbf{Lưu trữ chứng từ:}
    \begin{itemize}
        \item Lưu trữ các chứng từ nhập kho một cách cẩn thận và có hệ thống để phục vụ cho việc đối chiếu và kiểm tra sau này.
    \end{itemize}
    
    \item \textbf{Báo cáo nhập kho (tùy theo quy trình):}
    \begin{itemize}
        \item Lập báo cáo nhập kho gửi cho các bộ phận liên quan (kế toán, mua hàng,...).
    \end{itemize}
\end{itemize}

Đây là các bước cơ bản trong nghiệp vụ nhập kho. Quy trình cụ thể có thể thay đổi tùy thuộc vào đặc điểm ngành nghề và quy mô của từng doanh nghiệp.

\end{document}