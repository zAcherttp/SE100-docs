\documentclass[../main.tex]{subfiles}
\begin{document}

\subsection{Nghiệp vụ Đảm bảo an toàn kho}

\begin{itemize}
    \item \textbf{Tuân thủ nghiêm ngặt quy định về phòng cháy chữa cháy (PCCC):}
    \begin{itemize}
        \item Lắp đặt đầy đủ hệ thống PCCC theo quy định của nhà nước, yêu cầu phòng cháy.
        \item Kiểm tra, bảo dưỡng định kỳ hệ thống PCCC để đảm bảo hoạt động tốt.
        \item Trang bị biển báo PCCC rõ ràng, dễ thấy.
        \item Xây dựng sơ đồ thoát hiểm chi tiết và đề chương định tập PCCC định kỳ cho nhân viên trong kho.
        \item Đảm bảo lối đi lại trong kho luôn thông thoáng, không bị cản trở.
        \item Cấm hút thuốc và sử dụng nguồn lửa trần trong khu vực kho.
        \item Có quy trình xử lý sự cố cháy nổ cụ thể.
    \end{itemize}
    
    \item \textbf{Đảm bảo an toàn lao động:}
    \begin{itemize}
        \item Cung cấp đầy đủ trang thiết bị bảo hộ lao động (giày bảo hộ, găng tay, mũ bảo hiểm, kính bảo hộ…) cho nhân viên kho và yêu cầu sử dụng đúng cách.
        \item Huấn luyện an toàn lao động cho nhân viên về các thao tác bốc xếp, di chuyển hàng hóa, sử dụng thiết bị nâng hạ an toàn.
        \item Kiểm tra định kỳ tình trạng hoạt động của các thiết bị nâng hạ (xe nâng, xe đẩy…) và bảo trì khi cần thiết.
        \item Đảm bảo hệ thống chiếu sáng đầy đủ trong kho.
        \item Sắp xếp hàng hóa gọn gàng, chắc chắn, tránh đổ vỡ gây tai nạn.
        \item Có biển cảnh báo nguy hiểm ở những khu vực cần thiết.
        \item Xây dựng quy trình xử lý tai nạn lao động.
    \end{itemize}
    
    \item \textbf{Kiểm soát an ninh kho:}
    \begin{itemize}
        \item Lắp đặt hệ thống camera giám sát ở các vị trí quan trọng trong và ngoài kho.
        \item Hạn chế người không có phận sự ra vào khu vực kho.
        \item Thực hiện kiểm tra người và phương tiện ra vào kho (nếu cần).
        \item Có quy trình quản lý chìa khóa kho chặt chẽ.
        \item Tổ chức tuần tra, canh gác (nếu cần).
        \item Xây dựng quy trình xử lý các tình huống xâm nhập trái phép.
    \end{itemize}
    
    \item \textbf{Đảm bảo an toàn cho hàng hóa:}
    \begin{itemize}
        \item Lưu trữ hàng hóa theo đúng yêu đều, kiến bảo quản yêu cầu (nhiệt độ, độ ẩm...).
        \item Sử dụng vật liệu đóng gói phù hợp để bảo vệ hàng hóa trong quả trình lưu trữ và vận chuyển nội bộ.
        \item Kiểm tra định kỳ tình trạng hàng hóa để phát hiện sớm các dấu hiệu hư hỏng.
        \item Có biện pháp phòng chống côn trùng, mối mọt, ẩm mốc.
        \item Xây dựng quy trình xử lý hàng hóa bị hư hỏng, quá hạn.
    \end{itemize}
    
    \item \textbf{Thực hiện kiểm tra an toàn định kỳ:}
    \begin{itemize}
        \item Tổ chức kiểm tra an toàn toàn diện hoặc theo chuyên đề định kỳ (ví dụ: kiểm tra PCCC, kiểm tra an toàn lao động).
        \item Lập biên bản ghi nhận kết quả kiểm tra và các kiến nghị khắc phục.
        \item Theo dõi việc thực hiện các biện pháp khắc phục.
    \end{itemize}
\end{itemize}

Nghiệp vụ đảm bảo an toàn kho là nốt cốt quan trọng giúp quản lý kho hiệu quả, giúp bảo vệ tính mạng và sức khỏe của nhân viên, tài sản của doanh nghiệp và chất lượng hàng hóa. Việc tuân thủ nghiêm ngặt các quy trình và thực hiện đầy đủ các nghiệp vụ này sẽ góp phần nâng cao uy tín và thương hiệu của doanh nghiệp và hiệu quả.

Như vậy, quản lý kho không chỉ là một bộ phận hỗ trợ mà là một chức năng chiến lược, tác động trực tiếp đến lợi nhuận và khả năng cạnh tranh của doanh nghiệp. Việc áp dụng các quy trình quản lý kho hiệu quả, từ nhập kho, lưu kho, xuất kho, kiểm kê, báo cáo đến quản lý vị trí, sẽ giúp doanh nghiệp tối ưu hóa nguồn lực, giảm thiểu rủi ro và đáp ứng tốt hơn nhu cầu ngày càng cao của thị trường. Đầu tư vào quản lý kho chính là đầu tư vào sự thành công lâu dài của doanh nghiệp.

\end{document}