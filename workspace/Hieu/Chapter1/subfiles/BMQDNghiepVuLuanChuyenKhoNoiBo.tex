\documentclass[../main.tex]{subfiles}
\begin{document}

\subsection{Nghiệp vụ Luân chuyển Kho Nội bộ}

\subsubsection{Biểu mẫu (BM)}

\begin{table}[H]
\centering
\begin{tabular}{|L{7cm}|L{7cm}|}
\hline
\multicolumn{2}{|c|}{\textbf{BM10: Phiếu Luân chuyển kho}} \\
\hline
Mã phiếu: ................................. & Người tạo: .................................... \\
\hline
Ngày điều chuyển: .................... & Trạng thái: [ ] Chờ xuất [ ] Đang chuyển [ ] Đã nhận \\
\hline
Kho xuất: ................................. & Kho nhập: .................................... \\
\hline
\multicolumn{2}{|l|}{\textbf{Chi tiết Hàng hóa luân chuyển:}} \\
\hline
\textit{Mã SP} & \textit{Lô/Serial (nếu có)} \\
\hline
............. & .................................. \\
\hline
............. & .................................. \\
\hline
\end{tabular}
\caption{Biểu mẫu phiếu luân chuyển kho}
\label{tab:bm10-phieu-luan-chuyen-kho}
\end{table}

\subsubsection{Quy định (QĐ)}

\textbf{QĐ4: Quy định về Luân chuyển kho}

\begin{itemize}
    \item Mọi hoạt động chuyển hàng giữa các kho, chi nhánh bắt buộc phải lập Phiếu Luân chuyển kho.
    \item Kho xuất thực hiện nghiệp vụ "Xuất kho" và Kho nhập thực hiện nghiệp vụ "Nhập kho" dựa trên cùng một Phiếu Luân chuyển.
    \item Hệ thống chỉ tự động cập nhật tồn kho (trừ kho xuất, cộng kho nhập) khi Kho nhập xác nhận "Đã nhận hàng".
\end{itemize}

\end{document}