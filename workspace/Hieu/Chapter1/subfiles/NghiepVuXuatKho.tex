\documentclass[../main.tex]{subfiles}
\begin{document}

\subsection{Nghiệp vụ xuất kho}

\begin{itemize}
    \item \textbf{Tiếp nhận yêu cầu xuất kho:}
    \begin{itemize}
        \item Nhận phiếu yêu cầu xuất kho từ các bộ phận liên quan (bán hàng, sản xuất,...).
        \item Kiểm tra tính hợp lệ và đầy đủ của các thông tin trên phiếu yêu cầu (mã hàng, số lượng, lý do xuất,...).
    \end{itemize}
    
    \item \textbf{Tìm kiếm và chuẩn bị hàng hóa:}
    \begin{itemize}
        \item Xác định vị trí lưu trữ của hàng hóa cần xuất trong kho.
        \item Lấy hàng ra khu vực theo đúng số lượng và chủng loại được yêu cầu.
        \item Kiểm tra lại chất lượng và quy cách của hàng hóa trước khi xuất.
    \end{itemize}
    
    \item \textbf{Cập nhật thông tin xuất kho:}
    \begin{itemize}
        \item Ghi nhận thông tin xuất kho vào sổ sách kho hoặc phần mềm quản lý kho.
        \item Cập nhật số lượng tồn kho.
        \item Ghi lại thông tin người nhận và thời gian xuất (nếu cần).
    \end{itemize}
    
    \item \textbf{Đóng gói và bàn giao hàng hóa:}
    \begin{itemize}
        \item Đóng gói hàng hóa cẩn thận (nếu cần) để đảm bảo an toàn trong quá trình vận chuyển.
        \item Bàn giao hàng hóa cho người nhận kèm theo các chứng từ liên quan (phiếu xuất kho, hóa đơn,...).
        \item Yêu cầu người nhận ký xác nhận đã nhận đủ hàng.
    \end{itemize}
    
    \item \textbf{Lưu trữ chứng từ:}
    \begin{itemize}
        \item Lưu trữ các chứng từ xuất kho một cách cẩn thận và có hệ thống để phục vụ cho việc đối chiếu và kiểm tra sau này.
    \end{itemize}
    
    \item \textbf{Báo cáo xuất kho (tùy theo quy trình):}
    \begin{itemize}
        \item Lập báo cáo xuất kho gửi cho các bộ phận liên quan (kế toán, bán hàng,...).
    \end{itemize}
\end{itemize}

Tương tự như nghiệp vụ nhập kho, đây là các bước cơ bản trong nghiệp vụ xuất kho. Quy trình chi tiết có thể điều chỉnh tùy theo đặc thù hoạt động của từng doanh nghiệp.

\end{document}