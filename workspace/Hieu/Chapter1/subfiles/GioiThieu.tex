\documentclass[../main.tex]{subfiles}
\begin{document}

\section{Giới thiệu}

Quản lý kho là quá trình \textbf{theo dõi, kiểm soát số lượng và vị trí} của hàng hóa lưu trữ trong kho, đảm bảo quá trình xuất nhập diễn ra \textbf{suôn sẻ và nhanh chóng}. Mục tiêu chính của quản lý kho là:

\begin{itemize}
    \item \textbf{Duy trì mức tồn kho tối ưu:} Đảm bảo có đủ hàng hóa để đáp ứng nhu cầu của khách hàng mà không bị tồn kho quá nhiều, gây lãng phí chi phí lưu trữ.
    \item \textbf{Tối ưu hóa chi phí:} Giảm thiểu các chi phí liên quan đến kho bãi, nhân lực, và hao hụt hàng hóa.
    \item \textbf{Nâng cao hiệu quả hoạt động:} Đảm bảo quá trình nhập, xuất, lưu trữ và tìm kiếm hàng hóa diễn ra nhanh chóng và chính xác.
    \item \textbf{Đảm bảo chất lượng hàng hóa:} Lưu trữ và bảo quản hàng hóa đúng cách để tránh hư hỏng, mất mát.
    \item \textbf{Cung cấp thông tin chính xác:} Nắm bắt được số lượng, vị trí và tình trạng của hàng hóa trong kho để đưa ra các quyết định kinh doanh kịp thời.
\end{itemize}

\subsection{Các hoạt động chính trong quản lý kho:}

\begin{itemize}
    \item \textbf{Nhập kho:} Tiếp nhận, kiểm tra và ghi nhận thông tin hàng hóa nhập kho.
    \item \textbf{Lưu trữ:} Sắp xếp hàng hóa khoa học trong kho để dễ dàng tìm kiếm và quản lý.
    \item \textbf{Xuất kho:} Lấy hàng theo yêu cầu và ghi nhận thông tin xuất kho.
    \item \textbf{Kiểm kê:} Kiểm tra đối chiếu số lượng hàng hóa thực tế với số liệu trên sổ sách hoặc phần mềm.
    \item \textbf{Báo cáo:} Lập các báo cáo về tình hình nhập, xuất, tồn kho.
    \item \textbf{Quản lý vị trí:} Theo dõi vị trí cụ thể của từng loại hàng hóa trong kho.
    \item \textbf{Đảm bảo an toàn:} Thực hiện các biện pháp phòng cháy chữa cháy và an toàn lao động trong kho.
\end{itemize}

Quản lý kho hiệu quả đóng vai trò \textbf{quan trọng} trong hoạt động kinh doanh của mọi doanh nghiệp, giúp \textbf{tối ưu hóa chi phí, nâng cao hiệu suất và đáp ứng tốt hơn nhu cầu của khách hàng}.


\subsection{Vai trò của quản lý kho}

Quản lý kho đóng vai trò vô cùng quan trọng trong hoạt động của mọi doanh nghiệp, đặc biệt là các doanh nghiệp sản xuất và thương mại. Dưới đây là những vai trò chính của quản lý kho:

\begin{itemize}
    \item \textbf{Đảm bảo nguồn cung ổng lên lưu:} Quản lý kho hiệu quả giúp duy trì mức tồn kho hợp lý, đảm bảo luôn có đủ hàng hóa để đáp ứng nhu cầu sản xuất và bán hàng, tránh tình trạng gián đoạn do thiếu hụt hàng hóa hoặc nguyên vật liệu.
    
    \item \textbf{Tối ưu hóa chi phí:}
    \begin{itemize}
        \item \textbf{Giảm chi phí lưu trữ:} Việc quản lý kho khoa học giúp tận dụng tối đa không gian lưu trữ, giảm chi phí thuê kho và các chi phí liên quan.
        \item \textbf{Giảm thiểu hao hụt, mất mát:} Kiểm soát chặt chẽ quá trình nhập xuất và lưu trữ hàng hóa giúp hạn chế tình trạng hàng hóa bị hư hỏng, bị mất cắp.
        \item \textbf{Tránh tồn kho quá mức:} Quản lý kho giúp doanh nghiệp tránh tình trạng tồn kho tồn kho quá mức, tránh lãng phí vốn và chi phí lưu trữ không cần thiết.
    \end{itemize}
    
    \item \textbf{Nâng cao hiệu quả hoạt động:}
    \begin{itemize}
        \item \textbf{Tăng tốc độ xử lý đơn hàng:} Việc sắp xếp hàng hóa khoa học và quản lý vị trí chính xác giúp rút ngắn thời gian tìm kiếm và xuất hàng, nânh chóng đáp ứng kịp thời yêu cầu của khách hàng.
        \item \textbf{Giảm thiểu sai sót và lãm việc:} Quy trình nhập xuất kho rõ ràng và hiệu quả giúp nhân viên thực hiện công việc một cách trơn tru, giảm thiểu thời gian và công sức.
    \end{itemize}
    
    \item \textbf{Cải thiện dịch vụ khách hàng:}
    \begin{itemize}
        \item \textbf{Đảm bảo giao hàng đúng hẹn:} Việc có đủ hàng trong kho và quy trình xuất kho nhanh chóng giúp doanh nghiệp giao hàng đúng thời hạn, nâng cao sự hài lòng của khách hàng, nâng cao sự hài lòng.
        \item \textbf{Giảm tỷ lệ đơn hàng bị hủy hoặc bị trễ:} Nhờ đó giúp doanh nghiệp linh hoạt hơn trong việc đáp ứng các đơn hàng gấp hoặc các yêu cầu đặc biệt của khách hàng.
    \end{itemize}
    
    \item \textbf{Cung cấp thông tin chính xác cho việc ra quyết định:}
    \begin{itemize}
        \item \textbf{Theo dõi tình hình tồn kho:} Báo cáo kho định kỳ cung cấp thông tin chi tiết về số lượng, giá trị hàng tồn kho của từng sản phẩm và tổng quát của toàn bộ kho hình thức tế.
        \item \textbf{Hỗ trợ các quyết định mua sắm:} Dựa trên số liệu về nhập xuất kho và dế đoán nhu cầu trong tương lai, giúp doanh nghiệp đưa ra quyết định mua sắm hợp lý và hiệu quả hoặc sản xuất và mua hàng hiệu quả hơn.
        \item \textbf{Đánh giá hiệu quả kinh doanh:} Số liệu về hàng tồn kho và vòng quay hàng tồn kho, chi phí lưu trữ, chi phí lưu trữ giúp doanh nghiệp đánh giá hiệu quả hoạt động của kỳ phân kho và đưa ra các biện pháp cải tiến.
    \end{itemize}
    
    \item \textbf{Đảm bảo chất lượng hàng hóa:} Quản lý kho đúng cách bao gồm việc tạo môi trường lưu trữ phù hợp, kiểm soát nhiệt độ, độ ẩm, ánh sáng, giúp bảo quản hàng hóa ở tình trạng tốt nhất, tránh hư hỏng do các yếu tố bên ngoài.
\end{itemize}

\end{document}