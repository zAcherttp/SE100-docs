\documentclass[../main.tex]{subfiles}
\begin{document}

\subsection{Nghiệp vụ báo cáo kho}

\begin{itemize}
    \item \textbf{Xác định nhu cầu báo cáo:}
    \begin{itemize}
        \item Xác định mục đích của báo cáo (ví dụ: theo dõi hiệu suất, đưa ra quyết định mua hàng, phân tích xu hướng tình hình tồn kho).
        \item Xác định đối tượng sử dụng báo cáo (ví dụ: quản lý kho, bộ phận kế toán, ban giám đốc).
        \item Xác định tần suất lập báo cáo (ví dụ: hàng ngày, hàng tuần, hàng tháng, hàng quý, hàng năm).
    \end{itemize}
    
    \item \textbf{Thu thập dữ liệu:}
    \begin{itemize}
        \item Thu thập các số liệu liên quan từ hệ thống quản lý kho, sổ sách, hoặc các bộ phận liên quan (ví dụ: số liệu nhập kho, xuất kho, tồn kho, điều chuyển, kiểm kê).
        \item Đảm bảo tính chính xác và đầy đủ của số liệu.
    \end{itemize}
    
    \item \textbf{Xử lý và phân tích dữ liệu:}
    \begin{itemize}
        \item Sắp xếp, tóm tắt các hợp số liệu theo yêu cầu của báo cáo.
        \item Thực hiện các phép tính và phân tích cần thiết (ví dụ: tính toán tỷ lệ tồn kho, vòng quay hàng tồn kho, giá trị tồn kho).
        \item Sử dụng các công cụ hỗ trợ (ví dụ: Excel, giao diện mẫu báo cáo) để trực quan hóa số liệu (biểu đồ, bảng biểu).
    \end{itemize}
    
    \item \textbf{Lập báo cáo:}
    \begin{itemize}
        \item Trình bày thông tin một cách rõ ràng, dễ hiểu và trực quan.
        \item Sử dụng ngôn ngữ phù hợp với đối tượng người đọc.
        \item Nếu báo cáo thông tin quan trọng và cần diễn giải lâu ý.
        \item Đưa ra các nhận xét, đánh giá và đề xuất (nếu cần).
    \end{itemize}
    
    \item \textbf{Phân phối báo cáo:}
    \begin{itemize}
        \item Gửi báo cáo đến các đối tượng liên quan theo đúng thời gian quy định.
        \item Sử dụng các kênh phân phối phù hợp (ví dụ: email, hệ thống nội bộ).
    \end{itemize}
    
    \item \textbf{Lưu trữ báo cáo:}
    \begin{itemize}
        \item Lưu trữ các báo cáo đã phát hành một cách có hệ thống để dễ dàng tra cứu và so sánh cho các mục đích phân tích sau này.
    \end{itemize}
\end{itemize}

\textbf{Một số loại báo cáo kho thường gặp:}

\begin{itemize}
    \item Báo cáo nhập xuất tồn kho.
    \item Báo cáo tồn kho tối thiểu/tối đa.
    \item Báo cáo hàng hóa sắp hết hạn/quá hạn.
    \item Báo cáo giá trị tồn kho.
    \item Báo cáo vòng quay hàng tồn kho.
    \item Báo cáo kiểm kê kho.
    \item Báo cáo hiệu suất hoạt động kho.
\end{itemize}

Nghiệp vụ báo cáo kho cung cấp thông tin quan trọng giúp doanh nghiệp đưa ra các quyết định đúng đắn về quản lý kho, tối ưu hóa tồn kho và nâng cao hiệu quả hoạt động.

\end{document}