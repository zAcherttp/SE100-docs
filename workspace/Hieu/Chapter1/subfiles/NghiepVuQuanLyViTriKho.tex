\documentclass[../main.tex]{subfiles}
\begin{document}

\subsection{Nghiệp vụ quản lý vị trí kho}

\begin{itemize}
    \item \textbf{Xác định hệ thống định vị:}
    \begin{itemize}
        \item Lựa chọn hệ thống định vị phù hợp với quy mô và đặc điểm kho (ví dụ: hệ thống theo khu vực, theo dãy kệ, theo ô, sử dụng mã vạch, QR code).
        \item Thiết kế sơ đồ kho chi tiết, phân chia các khu vực, dãy, ô kệ để dễ dàng quản lý và tìm kiếm hàng hóa.
    \end{itemize}
    
    \item \textbf{Gán vị trí cho hàng hóa:}
    \begin{itemize}
        \item Khi hàng hóa nhập kho, xác định vị trí lưu trữ phù hợp dựa trên các tiêu chí (ví dụ: loại hàng, kích thước, tần suất xuất nhập).
        \item Ghi nhận vị trí lưu trữ vào hệ thống quản lý kho hoặc sổ sách.
        \item Dán nhãn hoặc mã hóa vị trí lên kệ hoặc ô chứa hàng.
    \end{itemize}
    
    \item \textbf{Cập nhật vị trí khi di chuyển hàng hóa:}
    \begin{itemize}
        \item Khi có sự thay đổi vị trí hàng hóa (ví dụ: chuyển từ khu vực lưu trữ tạm thời sang khu vực tối ưu chính, di chuyển để tối ưu không gian), cần cập nhật ngay thông tin vị trí trong hệ thống.
    \end{itemize}
    
    \item \textbf{Theo dõi và quản lý sơ đồ kho:}
    \begin{itemize}
        \item Duy trì sơ đồ kho luôn được cập nhật với vị trí hiện tại của hàng hóa.
        \item Sử dụng sơ đồ kho để dễ dàng tìm kiếm và xác định vị trí hàng hóa khi cần xuất kho hoặc thực hiện các hoạt động khác.
    \end{itemize}
    
    \item \textbf{Tối ưu hóa việc bố trí vị trí:}
    \begin{itemize}
        \item Phân tích tần suất xuất nhập của từng loại hàng hóa để bố trí vị trí lưu trữ hợp lý (hàng xuất nhập thường xuyên nên đặt ở vị trí dễ tiếp cận).
        \item Cân nhắc đến đặc tính của hàng hóa (ví dụ: hàng dễ vỡ, hàng cần bảo quản đặc biệt) khi lựa chọn vị trí.
    \end{itemize}
    
    \item \textbf{Kiểm tra và đối chiếu vị trí:}
    \begin{itemize}
        \item Định kỳ kiểm tra vị trí thực tế của các thông tin vị trí trong hệ thống so với vị trí thực tế của hàng hóa trong kho.
        \item Điều chỉnh các sai lệch nếu có.
    \end{itemize}
    
    \item \textbf{Đào tạo nhân viên:}
    \begin{itemize}
        \item Đào tạo nhân viên kho về hệ thống quản lý vị trí và các quy trình liên quan.
        \item Đảm bảo nhân viên hiểu rõ cách sử dụng sơ đồ kho, tác đơn nhập, xuất và bảng vị trí hàng hóa.
    \end{itemize}
\end{itemize}

Quản lý vị trí trong kho hiệu quả giúp tăng tốc độ tìm kiếm và xuất nhập hàng hóa, giảm thiểu sai sót, tối ưu hóa không gian lưu trữ và nâng cao hiệu quả hoạt động chung của kho.

\end{document}