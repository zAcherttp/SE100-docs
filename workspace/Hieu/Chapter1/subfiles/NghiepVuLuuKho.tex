\documentclass[../main.tex]{subfiles}
\begin{document}

\subsection{Nghiệp vụ lưu kho}

\begin{itemize}
    \item \textbf{Sắp xếp và bố trí hàng hóa:}
    \begin{itemize}
        \item Sắp xếp hàng hóa trong kho một cách khoa học, có hệ thống, đảm bảo dễ dàng tìm kiếm, di chuyển và kiểm kê.
        \item Tối ưu hóa không gian lưu trữ để chứa được nhiều hàng hóa nhất có thể.
        \item Phân loại hàng hóa theo các tiêu chí phù hợp (ví dụ: loại sản phẩm, lô sản xuất, hạn sử dụng).
        \item Sử dụng các thiết bị hỗ trợ (kệ, giá đỡ, pallet...) một cách hiệu quả và an toàn.
    \end{itemize}
    
    \item \textbf{Bảo quản hàng hóa:}
    \begin{itemize}
        \item Đảm bảo các điều kiện lưu trữ phù hợp với từng loại hàng hóa (nhiệt độ, độ ẩm, ánh sáng...).
        \item Thực hiện các biện pháp phòng chống côn trùng, mối mọt, ẩm mốc.
        \item Kiểm tra định kỳ tình trạng hàng hóa để phát hiện và xử lý kịp thời các vấn đề phát sinh.
    \end{itemize}
    
    \item \textbf{Quản lý vị trí lưu trữ:}
    \begin{itemize}
        \item Ghi chép và cập nhật chính xác vị trí của từng loại hàng hóa trong kho.
        \item Sử dụng hệ thống mã vạch hoặc các phương pháp định vị khác để dễ dàng xác định vị trí.
    \end{itemize}
    
    \item \textbf{Đảm bảo an toàn kho bãi:}
    \begin{itemize}
        \item Tuân thủ các quy định về phòng cháy chữa cháy.
        \item Đảm bảo hệ thống chiếu sáng, thông gió đầy đủ.
        \item Trang bị các thiết bị bảo hộ lao động cho nhân viên kho.
        \item Thực hiện kiểm tra an toàn định kỳ.
    \end{itemize}
    
    \item \textbf{Kiểm kê định kỳ:}
    \begin{itemize}
        \item Tiến hành kiểm kê hàng hóa theo định kỳ (ngày, tuần, tháng, quý, năm) để đối chiếu số lượng thực tế với số liệu trên sổ sách hoặc phần mềm.
        \item Xác định các sai lệch và tìm ra nguyên nhân để có biện pháp xử lý.
    \end{itemize}
    
    \item \textbf{Theo dõi và đánh giá hiệu quả lưu kho:}
    \begin{itemize}
        \item Theo dõi các chỉ số liên quan đến hoạt động lưu kho (ví dụ: tỷ lệ sử dụng không gian, chi phí lưu trữ, tần suất di chuyển hàng hóa).
        \item Đánh giá hiệu quả của các quy trình lưu kho và đề xuất các cải tiến cần thiết.
    \end{itemize}
\end{itemize}

Các nghiệp vụ lưu kho này đóng vai trò then chốt trong việc duy trì chất lượng hàng hóa, tối ưu hóa không gian và chi phí, đồng thời đảm bảo hoạt động xuất nhập diễn ra suôn sẻ.

\end{document}