\documentclass[../main.tex]{subfiles}
\begin{document}

\subsection{Đối tượng hướng đến}

Dựa trên các phân tích về thị trường và các quy mô doanh nghiệp trong tài liệu, hệ thống quản lý kho này được thiết kế để tập trung vào các đối tượng khách hàng cụ thể sau:

\begin{itemize}
    \item \textbf{Doanh nghiệp nhỏ lẻ và các chuỗi cửa hàng (SMEs \& Chains):}
    \begin{itemize}
        \item Đây là phân khúc đã vượt qua quy mô quản lý bằng sổ sách hoặc các tính năng quản lý kho cơ bản tích hợp sẵn trên phần mềm bán hàng (POS).
        \item Những doanh nghiệp này cần một giải pháp dễ chuẩn hóa quy trình, quản lý đa kho, đa chi nhánh, nhưng không yêu cầu sự phức tạp và chi phí triển khai không lồ như các hệ thống WMS/ERP cấp tập đoàn.
        \item Đối tượng này bao gồm các công ty phân phối nhỏ, các doanh nghiệp kinh doanh trên sàn thương mại điện tử (cần xử lý đơn hàng nhanh), và các mô hình chuỗi cửa hàng (từ một đến 20+ chi nhánh).
    \end{itemize}
    
    \item \textbf{Đáp ứng các loại mặt hàng linh hoạt, bao phủ nhiều lĩnh vực:}
    \begin{itemize}
        \item Hệ thống được thiết kế với tính linh hoạt cao, không bị giới hạn trong một ngành hàng cụ thể.
        \item Khả năng tùy chỉnh thông tin và thuộc tính sản phẩm cho phép hệ thống đáp ứng nhu cầu của nhiều lĩnh vực khác nhau như:
        \begin{itemize}
            \item \textbf{Thời trang:} Quản lý theo mẫu, màu sắc, kích cỡ (size).
            \item \textbf{Bán lẻ/Tạp hóa (FMCG):} Quản lý theo lô sản xuất và hạn sử dụng.
            \item \textbf{Điện tử/Thiết bị:} Quản lý theo số sê-ri (Serial Number) hoặc IMEI.
            \item \textbf{Dược phẩm/Mỹ phẩm:} Quản lý nghiêm ngặt theo lô, hạn sử dụng, và số đăng ký.
            \item \textbf{Vật tư/Linh kiện:} Quản lý theo vị trí chi tiết (ô, kệ) để tối ưu thời gian tìm kiếm.
        \end{itemize}
    \end{itemize}
\end{itemize}

\end{document}