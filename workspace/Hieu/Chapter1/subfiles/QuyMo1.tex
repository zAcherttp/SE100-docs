\documentclass[../main.tex]{subfiles}
\begin{document}

\textbf{QUY MÔ 1: HỘ KINH DOANH, CỬA HÀNG NHỎ LẺ}

\begin{itemize}
    \item \textbf{Phần mềm tiêu biểu:} KiotViet
    \item \textbf{Đối tượng và Nhu cầu cốt lõi:}
    \begin{itemize}
        \item \textbf{Đối tượng:} Các cửa hàng tạp hóa, shop thời trang, quán cà phê, hiệu thuốc nhỏ.
        \item \textbf{Quy mô:} Thường do 1 chủ quản lý, có từ 1-5 nhân viên. Lượng hàng hóa không quá lớn.
        \item \textbf{Nhu cầu:} Cần một công cụ \textbf{đơn giản, dễ sử dụng, chi phí thấp} để thay thế sổ sách. Yêu cầu quan trọng nhất là quản lý bán hàng (POS) phải được tích hợp chặt chẽ với quản lý tồn kho. Họ không cần các quy trình kho phức tạp như quản lý vị trí hay tối ưu lộ trình.
    \end{itemize}
\end{itemize}


\begin{itemize}
    \item \textbf{Phân tích chi tiết: KiotViet đáp ứng nhu cầu như thế nào?}
\end{itemize}

\begin{table}[H]
\centering
\begin{tabular}{|L{3.5cm}|L{5cm}|L{5.5cm}|}
\hline
\rowcolor{headerblue}
\textcolor{white}{\textbf{Tính năng của KiotViet}} & \textcolor{white}{\textbf{Vấn đề của Cửa hàng}} & \textcolor{white}{\textbf{Giải pháp của KiotViet}} \\
\hline
\textbf{Tích hợp Quản lý Bán hàng} & Chủ cửa hàng vừa phải bán hàng, vừa phải quản lý kho. Dùng 2 phần mềm riêng biệt hoặc ghi sổ rất mất thời gian và dễ sai sót. & Khi một sản phẩm được bán qua giao diện POS, số lượng tồn kho sẽ tự \textbf{động được trừ đi trong thời gian thực}. Điều này giúp chủ cửa hàng luôn biết chính xác lượng hàng còn lại mà không cần tính tay, tính thủ công. \\
\hline
\textbf{Quản lý hàng hóa đơn giản} & Hàng hóa có nhiều thuộc tính (ví dụ: áo có size S, M, L; màu 50, xanh). Quản lý bằng sổ sách rất rối và khó tìm kiếm. & Cho phép tạo sản phẩm với nhiều thuộc tính, mỗi thuộc tính có mã hàng và số lượng tồn kho riêng, giúp theo dõi việc cập nhật và tìm kiếm hàng hóa trở nên nhanh chóng, chính xác. \\
\hline
\textbf{Cảnh báo tồn kho tối thiểu} & Chủ cửa hàng bận rộn dễ quên nhập hàng, dẫn đến "cháy hàng" các sản phẩm bán chạy, làm mất khách. & Cho phép thiết lập "định mức tồn kho tối thiểu". Khi số lượng tồn kho giảm xuống dưới mức này, phần mềm sẽ tự động cảnh báo, giúp chủ cửa hàng chủ động lên kế hoạch nhập hàng mới. \\
\hline
\textbf{Sử dụng trên nhiều thiết bị} & Chủ cửa hàng thường xuyên đi chuyển, không thể lúc nào cũng ngồi ở cửa hàng để kiểm tra số liệu. & Cung cấp ứng dụng di động cho phép chủ cửa hàng xem báo cáo bán hàng, kiểm tra tồn kho, tạo đơn nhập hàng... ở bất cứ đâu, bất cứ lúc nào. \\
\hline
\textbf{Chi phí hợp lý} & Ngân sách cho công nghệ của các hộ kinh doanh rất hạn chế, khó đầu tư một khoản tiền lớn ban đầu. & Cung cấp mô hình thuê bao theo tháng với chi phí thấp, giúp họ dễ dàng tiếp cận công nghệ mà không cần đầu tư lớn. \\
\hline
\end{tabular}
\caption{Phân tích chi tiết: KiotViet đáp ứng nhu cầu của cửa hàng nhỏ lẻ}
\label{tab:kiotviet-analysis}
\end{table}

\textbf{Kết luận:} KiotViet là lựa chọn hoàn hảo cho quy mô này vì nó \textbf{không có gắng làm những gì phức tạp}. Phần mềm tập trung giải quyết đúng và đủ bài toán cốt lõi: \textbf{kết hợp bán hàng và quản lý tồn kho một cách đơn giản, trực quan và hiệu quả nhất} với chi phí tối thiểu.


\end{document}