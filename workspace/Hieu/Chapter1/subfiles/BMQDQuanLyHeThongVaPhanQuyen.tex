\documentclass[../main.tex]{subfiles}
\begin{document}

\subsection{Quản lý Hệ thống và Phân quyền}

\subsubsection{Biểu mẫu (BM)}

\begin{table}[H]
\centering
\begin{tabular}{|L{7cm}|L{7cm}|}
\hline
\multicolumn{2}{|c|}{\textbf{BM1: Khởi tạo Workspace}} \\
\hline
Tên công ty/cửa hàng: ..................... & Địa chỉ: ........................................ \\
\hline
Quản trị viên (Admin): .................... & Email Admin: ................................. \\
\hline
\multicolumn{2}{|c|}{\textbf{Danh sách kho hàng (Cấu hình)}} \\
\hline
Mã kho 1: ................................. & Tên kho 1: ................................... \\
\hline
Mã kho 2: ................................. & Tên kho 2: ................................... \\
\hline
\end{tabular}
\caption{Biểu mẫu khởi tạo Workspace}
\label{tab:bm1-khoi-tao-workspace}
\end{table}

\begin{table}[H]
\centering
\begin{tabular}{|L{7cm}|L{7cm}|}
\hline
\multicolumn{2}{|c|}{\textbf{BM2: Quản lý Nhân viên}} \\
\hline
Mã nhân viên: ............................. & Họ và tên: ...................................... \\
\hline
Tên đăng nhập: .......................... & Mật khẩu (khởi tạo): ........................ \\
\hline
Email / SĐT: .............................. & Vai trò: ........................................ \\
\hline
Kho/Chi nhánh làm việc: .............. & Trạng thái: [ ] Hoạt động [ ] Khóa \\
\hline
\end{tabular}
\caption{Biểu mẫu quản lý nhân viên}
\label{tab:bm2-quan-ly-nhan-vien}
\end{table}

\begin{table}[H]
\centering
\begin{tabular}{|L{7cm}|L{7cm}|}
\hline
\multicolumn{2}{|c|}{\textbf{BM3: Quản lý Vai trò (Phân quyền)}} \\
\hline
Tên vai trò: ................................ & Mô tả vai trò: ................................ \\
\hline
\multicolumn{2}{|l|}{\textbf{Chi tiết quyền hạn (Checklist):}} \\
\hline
[ ] Module Nhập kho (Xem, Tạo, Duyệt) & [ ] Module Xuất kho (Xem, Tạo, Duyệt) \\
\hline
[ ] Module Kiểm kê (Xem, Tạo, Duyệt) & [ ] Module Báo cáo (Xem) \\
\hline
[ ] Module Cài đặt (Quản lý) & [ ] Module Luân chuyển (Xem, Tạo) \\
\hline
\end{tabular}
\caption{Biểu mẫu quản lý vai trò (Phân quyền)}
\label{tab:bm3-quan-ly-vai-tro}
\end{table}

\subsubsection{Quy định (QĐ)}

\textbf{QĐ1: Quy định về Quản lý hệ thống}

\begin{itemize}
    \item Chỉ Quản trị viên (Admin) mới có quyền tạo/khóa tài khoản nhân viên và thiết lập Vai trò (phân quyền).
    \item Việc gán quyền phải tuân thủ đúng chức năng, nhiệm vụ của từng nhân viên (ví dụ: nhân viên bán hàng chỉ xem tồn kho để tạo đơn hàng, không được nhập/xuất kho).
    \item Nghiêm cấm việc sử dụng chung tài khoản hoặc tiết lộ mật khẩu.
\end{itemize}

\end{document}