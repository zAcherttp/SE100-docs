\documentclass[../main.tex]{subfiles}
\begin{document}

\subsection{Quy trình sử dụng}

Quy trình vận hành phần mềm quản lý kho được chia thành hai giai đoạn chính: Cài đặt ban đầu và Vận hành hệ thống.

\subsubsection{Giai đoạn Cài đặt}

Đây là các bước nền tảng để hệ thống sẵn sàng đi vào hoạt động, thường chỉ thực hiện một lần khi bắt đầu sử dụng:

\begin{itemize}
    \item \textbf{Khởi tạo Workspace:} Quản trị viên (chủ doanh nghiệp hoặc người quản lý chính) tạo workspace - không gian làm việc độc lập. Tại đây, người dùng sẽ thiết lập các thông tin cơ bản như tên công ty hoặc hàng. Gia đit và các hình các kho hàng (ví dụ: kho chính, kho chi nhánh A, kho chi nhánh B).
    \item \textbf{Thêm nhân viên:} Tạo các tài khoản người dùng cho toàn bộ nhân viên sẽ thao tác trên hệ thống (ví dụ: quản lý kho, thủ kho, nhân viên bán hàng, kế toán).
    \item \textbf{Phân quyền:} Gán quyền hạn cụ thể cho từng vai trò hoặc nhóm nhân viên. Ví dụ: nhân viên bán hàng chỉ có thể xem tồn kho để tạo đơn hàng, thủ kho được phép thực hiện nhập/xuất hàng, quản lý kho có toàn quyền (quản lý vị trí, sắp xếp hàng hóa, kiểm kê), thống cô thể thắng mọi dụng thông qua vắóc bảo và dữ để gắn các thuông hản quyền hạn cho phép quản lý các quyên ô mức chi tiết (granular).
\end{itemize}

\subsubsection{Giai đoạn Vận hành}

Đây là các hoạt động nghiệp vụ cơ chỉ lặp lại hàng ngày để quản lý dông của của hàng hóa:

\begin{itemize}
    \item \textbf{Nhập kho:}
    \begin{enumerate}
        \item Tạo phiếu nhập kho.
        \item Nhập đủ thông tin nhập phiếu về số đường nhập ký (như nhớn thoại/mẩi quyết mắ sản phấm, số lượng, giá, nhà cung cấp).
        \item Hệ thống xác nhận số lượng, ghi nhận lộ/hạn sử dụng (nếu có) và cập nhật tồn kho tròn thời gian thực. Hệ thống có thơ gọi ý về lúu trữ (ví dụ: Ô-Kê nào).
    \end{enumerate}
    
    \item \textbf{Xuất kho:}
    \begin{enumerate}
        \item Tạo phiếu xuất kho.
        \item Chọn sản phẩm cần lấy hàng (picking list) và chi dẫn nhân viên kho đi trình đi lấy hàng nạnh nhất trong kho.
        \item Nhân viên lấy hàng và quét mã vạch sản phẩm để xác nhận. Hệ thống đổ chính xác trạm giao xả hàng.
        \item Số lượng tồn kho để thống tự đông trụ tồn kho.
    \end{enumerate}
    
    \item \textbf{Kiểm kê:}
    \begin{enumerate}
        \item Chọn và tạo một phiên kiểm kê.
        \item Quét mã vạch/nhập liệu số lượng đệ quết mả vạch và điễm số lương thực tế.
        \item Hệ thống tự động đối chiếu ngay lập túc số đều thức tế với số liệu trên để sách và hiển thi các mắt hàng đỗ chọnh lệch.
        \item Quản lý các nhận viết cửa và hệ thống tư động tạo các phiếu điều chỉnh để đưa số liệu tồn kho về đúng với thực tế.
    \end{enumerate}
\end{itemize}

\end{document}