\documentclass[../main.tex]{subfiles}

\begin{document}

\begin{table}[h]
\centering
\caption{Bảng UC02-01}
\begin{tabular}{|p{4cm}|p{10cm}|}
\hline
\textbf{Mã UC} & UC02-01 \\
\hline
\textbf{Tên UC} & Create Category (Tạo Danh mục) \\
\hline
\textbf{Mô tả tóm tắt} & Cho phép quản trị viên tạo một danh mục sản phẩm mới (có phân cấp). \\
\hline
\textbf{Tác nhân tham gia} & Quản lý kho / Quản trị viên hàng tồn kho \\
\hline
\textbf{Luồng sự kiện chính} & 
\begin{enumerate}
    \vspace{-18pt}
    \item Tác nhân điều hướng đến "Dữ liệu chủ" -> "Danh mục".
    \item Tác nhân chọn "Tạo Danh mục mới".
    \item Tác nhân nhập tên danh mục.
    \item (Tùy chọn) Tác nhân chọn một danh mục cha.
    \item Tác nhân nhấn "Lưu".
    \item Hệ thống «include» Use ltree for Hierarchy (tạo đường dẫn ltree, ví dụ: 'Electronics.Laptops').
    \item Hệ thống lưu danh mục mới vào CSDL.
\end{enumerate} \\
\hline
\textbf{Luồng sự kiện khác} & 
\begin{enumerate}
    \vspace{-18pt}
    \item[5a.] Tên danh mục trống hoặc trùng lặp trong cùng cấp cha -> Hệ thống hiển thị lỗi.
\end{enumerate} \\
\hline
\textbf{Yêu cầu đặc biệt} & Cần sử dụng PostgreSQL ltree extension để quản lý phân cấp. \\
\hline
\textbf{Trạng thái hệ thống ngay khi bắt đầu thực hiện} & Tác nhân đã đăng nhập và có quyền "MANAGE_CATEGORIES". \\
\hline
\textbf{Trạng thái hệ thống sau khi bắt đầu thực hiện} & Một danh mục mới được thêm vào hệ thống. \\
\hline
\textbf{Điểm mở rộng} & Không có \\
\hline
\end{tabular}
\end{table}

\end{document}