\documentclass[\subfix{../../main.tex}]{subfiles}

\begin{document}

\begin{table}[h]
\centering
\begin{tabular}{|p{4cm}|p{10cm}|}
\hline
\textbf{Mã UC} & UC09-01 \\
\hline
\textbf{Tên UC} & Tạo Đơn Chuyển kho (Create Transfer Order) \\
\hline
\textbf{Mô tả tóm tắt} & Lập đơn yêu cầu chuyển hàng hóa từ một chi nhánh (kho nguồn) sang một chi nhánh khác (kho đích). \\
\hline
\textbf{Tác nhân tham gia} & Inventory Administrator (QT Tồn kho), Warehouse Manager (Quản lý Kho) \\
\hline
\textbf{Sự kiện kích hoạt} & Người dùng chọn chức năng "Tạo Đơn Chuyển kho". \\
\hline
\textbf{Luồng sự kiện chính} & 
\begin{enumerate}
    \vspace{-18pt}
    \item Hệ thống tạo Mã Chuyển kho (Transfer Code).
    \item Người dùng chọn Kho Nguồn (Source Branch).
    \item Người dùng chọn Kho Đích (Destination Branch).
    \item Người dùng thêm các mặt hàng (SKU, Số lượng) cần chuyển.
    \item Người dùng nhập Ngày giao hàng dự kiến.
    \item Người dùng lưu đơn, hệ thống đặt trạng thái "TRANSFER_DRAFT".
\end{enumerate} \\
\hline
\textbf{Luồng sự kiện khác} & Kiểm tra tồn kho nguồn: khi thêm SKU, hệ thống kiểm tra tồn kho khả dụng tại Kho Nguồn; nếu không đủ, cảnh báo nhưng vẫn cho phép tạo đơn. \\
\hline
\textbf{Yêu cầu trước khi thực hiện} & Người dùng đã đăng nhập. \\
\hline
\textbf{Yêu cầu sau khi thực hiện} & Một đơn chuyển kho được tạo ở trạng thái "TRANSFER_DRAFT". \\
\hline
\textbf{Yêu cầu phi chức năng} & (Không có) \\
\hline
\end{tabular}

\caption{Bảng UC09-01}
\end{table}

\end{document}
