\documentclass[\subfix{../../main.tex}]{subfiles}

\begin{document}

\begin{table}[h]
\centering
\caption{Đặc tả use case}
\begin{tabular}{|p{4cm}|p{10cm}|}
\hline
\textbf{Mã UC} & UC10-04 \\
\hline
\textbf{Tên UC} & Tạo Mẫu Báo cáo Tùy chỉnh (Create Custom Report Template) \\
\hline
\textbf{Mô tả tóm tắt} & Thiết kế và lưu lại một mẫu báo cáo mới theo nhu cầu riêng. \\
\hline
\textbf{Tác nhân tham gia} & Warehouse Manager (Quản lý Kho), Accountant (Kế toán) \\
\hline
\textbf{Sự kiện kích hoạt} & Người dùng chọn chức năng "Thiết kế Báo cáo Mới". \\
\hline
\textbf{Luồng sự kiện chính} & 
\begin{enumerate}
    \vspace{-18pt}
    \item Người dùng chọn nguồn dữ liệu (VD: Tồn kho, Giao dịch, Đơn hàng...).
    \item Người dùng kéo/thả các cột muốn hiển thị và định nghĩa bộ lọc, nhóm, tổng hợp.
    \item Người dùng đặt tên, lưu mẫu và thiết lập quyền chia sẻ.
\end{enumerate} \\
\hline
\textbf{Luồng sự kiện khác} & (Không có) \\
\hline
\textbf{Yêu cầu trước khi thực hiện} & Người dùng đã đăng nhập và có quyền "CREATE_CUSTOM_REPORTS". \\
\hline
\textbf{Yêu cầu sau khi thực hiện} & Một mẫu báo cáo mới được lưu lại. \\
\hline
\textbf{Yêu cầu phi chức năng} & Giao diện thiết kế phải trực quan (kéo-thả). \\
\hline
\end{tabular}
\end{table}

\end{document}
