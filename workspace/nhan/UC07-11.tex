\documentclass[\subfix{../../main.tex}]{subfiles}

\begin{document}

\begin{table}[h]
\centering
\caption{Bảng UC07-11}
\begin{tabular}{|p{4cm}|p{10cm}|}
\hline
\textbf{Mã UC} & UC07-11 \\
\hline
\textbf{Tên UC} & Cảnh báo Tồn kho thấp (Hệ thống) \\
\hline
\textbf{Mô tả tóm tắt} & Hệ thống tự động cảnh báo khi tồn kho của SKU giảm dưới mức tối thiểu (ROP). \\
\hline
\textbf{Tác nhân tham gia} & System (Hệ thống) \\
\hline
\textbf{Sự kiện kích hoạt} & Kích hoạt mỗi khi có giao dịch làm giảm tồn kho (xuất kho, điều chỉnh giảm). \\
\hline
\textbf{Luồng sự kiện chính} & 
\begin{enumerate}
    \vspace{-18pt}
    \item Một giao dịch làm thay đổi số lượng tồn kho của SKU.
    \item Hệ thống so sánh tổng tồn kho hiện tại với Điểm Tái đặt hàng (ROP) của SKU đó.
    \item Nếu Tồn kho <= ROP, hệ thống tạo một thông báo (UC11-06).
    \item Gửi thông báo đến Bộ phận Mua hàng.
\end{enumerate} \\
\hline
\textbf{Luồng sự kiện khác} & (Không có) \\
\hline
\textbf{Yêu cầu trước khi thực hiện} & Một sự kiện (transaction) làm thay đổi (giảm) tồn kho.\\
\hline
\textbf{Yêu cầu sau khi thực hiện} & Thông báo tồn kho thấp được tạo ra. \\
\hline
\textbf{Yêu cầu phi chức năng} & Việc kiểm tra phải diễn ra tức thì (real-time) sau giao dịch. \\
\hline
\end{tabular}
\end{table}

\end{document}
