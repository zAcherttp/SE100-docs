\documentclass[\subfix{../../main.tex}]{subfiles}

\begin{document}

\begin{table}[h]
\centering
\caption{Bảng UC08-01}
\begin{tabular}{|p{4cm}|p{10cm}|}
\hline
\textbf{Mã UC} & UC08-01 \\
\hline
\textbf{Tên UC} & Tạo Yêu cầu Trả hàng (Create Return Request) \\
\hline
\textbf{Mô tả tóm tắt} & Lập một yêu cầu (phiếu) để trả hàng lại cho nhà cung cấp. \\
\hline
\textbf{Tác nhân tham gia} & Inventory Administrator (QT Tồn kho), Stock Clerk (NV Kho) \\
\hline
\textbf{Sự kiện kích hoạt} & Người dùng chọn chức năng "Tạo Yêu cầu Trả hàng". \\
\hline
\textbf{Luồng sự kiện chính} & 
\begin{enumerate}
    \vspace{-18pt}
    \item Hệ thống tạo mã yêu cầu trả hàng (Return Request Code).
    \item Người dùng chọn Nhà cung cấp.
    \item Người dùng thêm (các) mặt hàng cần trả (chọn SKU, Lô, Số lượng).
    \item Với mỗi mặt hàng, người dùng chọn "Lý do Trả" (VD: DEFECTIVE, EXPIRED...).
    \item Người dùng nhập ghi chú, đính kèm bằng chứng (ảnh) và nhập số tiền tín dụng dự kiến (nếu có).
    \item Người dùng lưu yêu cầu, hệ thống đặt trạng thái "RETURN_DRAFT".
\end{enumerate} \\
\hline
\textbf{Luồng sự kiện khác} & Lý do "KHÁC": Người dùng chọn "OTHER" và hệ thống yêu cầu ghi chú chi tiết. \\
\hline
\textbf{Yêu cầu trước khi thực hiện} & Người dùng đã đăng nhập.\\
\hline
\textbf{Yêu cầu sau khi thực hiện} & Một yêu cầu trả hàng được tạo ở trạng thái "RETURN_DRAFT". \\
\hline
\textbf{Yêu cầu phi chức năng} & Hỗ trợ tải lên tối đa 5 tệp bằng chứng (ảnh/video). \\
\hline
\end{tabular}
\end{table}

\end{document}
