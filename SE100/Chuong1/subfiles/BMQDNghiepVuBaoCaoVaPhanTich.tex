\documentclass[../chapter1.tex]{subfiles}
\begin{document}

\subsection{Nghiệp vụ Báo cáo \& Phân tích}

\subsubsection{Biểu mẫu (BM)}

\textit{(Nghiệp vụ báo cáo là kết quả đầu ra của hệ thống, không phải là biểu mẫu nhập liệu. Dưới đây là các quy định về báo cáo.)}

\subsubsection{Quy định (QĐ)}

\textbf{QĐ7: Quy định về Báo cáo}

\begin{itemize}
    \item Hệ thống phải cung cấp các loại báo cáo tiêu chuẩn: Báo cáo Nhập-Xuất-Tồn, Báo cáo Giá trị Tồn kho, Báo cáo Kiểm kê.
    \item Hệ thống phải tự động cập nhật các Báo cáo phân tích thông minh: Báo cáo hàng tồn kho chậm luân chuyển, Báo cáo hàng bán chạy, Báo cáo hàng sắp hết hạn/quá hạn, Báo cáo hiệu suất vận hành (thời gian xử lý đơn).
    \item Quản lý kho có trách nhiệm xem xét các báo cáo (đặc biệt là hàng chậm luân chuyển, hàng sắp hết hạn) hàng tuần để đề xuất xử lý.
    \item Bộ phận Kế toán sử dụng Báo cáo Giá trị Tồn kho để đối soát tài chính.
\end{itemize}

\end{document}