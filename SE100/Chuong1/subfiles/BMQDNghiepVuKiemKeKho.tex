\documentclass[../chapter1.tex]{subfiles}
\begin{document}

\subsection{Nghiệp vụ Kiểm kê Kho}

\subsubsection{Biểu mẫu (BM)}

\begin{table}[H]
\centering
\begin{tabular}{|L{7cm}|L{7cm}|}
\hline
\multicolumn{2}{|c|}{\textbf{BM11: Phiên Kiểm kê}} \\
\hline
Mã phiên: ................................. & Ngày tạo: .................................... \\
\hline
Người tạo: ................................ & Trạng thái: [ ] Mới tạo [ ] Đang kiểm [ ] Chờ xử lý [ ] Hoàn thành \\
\hline
Loại kiểm kê: [ ] Toàn phần [ ] Theo chu kỳ & Phạm vi (Kho, Khu vực, Nhóm hàng): ... \\
\hline
\multicolumn{2}{|l|}{Nhân viên thực hiện: ...................} \\
\hline
\end{tabular}
\caption{Biểu mẫu phiên kiểm kê}
\label{tab:bm11-phien-kiem-ke}
\end{table}

\begin{table}[H]
\centering
\begin{tabular}{|L{7cm}|L{7cm}|}
\hline
\multicolumn{2}{|c|}{\textbf{BM12: Phiếu Đếm hàng (Trên Mobile App)}} \\
\hline
Mã SP (Quét): .......................... & Tên SP: ....................................... \\
\hline
Vị trí (Khu/Kệ/Ô): .................... & Tồn kho sổ sách: .......................... \\
\hline
Số lượng thực tế (Đếm): .......... & Lô / HSD (nếu có): ....................... \\
\hline
\multicolumn{2}{|l|}{Ghi chú: ...................................} \\
\hline
\end{tabular}
\caption{Biểu mẫu phiếu đếm hàng (Trên Mobile App)}
\label{tab:bm12-phieu-dem-hang}
\end{table}

\begin{table}[H]
\centering
\begin{tabular}{|L{7cm}|L{7cm}|}
\hline
\multicolumn{2}{|c|}{\textbf{BM13: Báo cáo Chênh lệch Kiểm kê}} \\
\hline
Mã phiên: ................................. & Ngày hoàn thành đếm: ..................... \\
\hline
\multicolumn{2}{|l|}{\textbf{Chi tiết chênh lệch:}} \\
\hline
\textit{Mã SP} & \textit{Tồn sổ sách} \\
\hline
............. & .................... \\
\hline
............. & .................... \\
\hline
\end{tabular}
\caption{Báo cáo chênh lệch kiểm kê}
\label{tab:bm13-bao-cao-chenh-lech}
\end{table}

\begin{table}[H]
\centering
\begin{tabular}{|L{7cm}|L{7cm}|}
\hline
\multicolumn{2}{|c|}{\textbf{BM14: Phiếu Điều chỉnh Kho}} \\
\hline
Mã phiếu: ................................. & Tham chiếu (Mã phiên kiểm kê): ......... \\
\hline
Lý do điều chỉnh: Chênh lệch kiểm kê & Người duyệt (Quản lý): ................... \\
\hline
\multicolumn{2}{|l|}{\textbf{Chi tiết điều chỉnh (Hệ thống tự động tạo):}} \\
\hline
\textit{Mã SP} & \textit{Điều chỉnh ( +/- )} \\
\hline
............. & .................................. \\
\hline
............. & .................................. \\
\hline
\end{tabular}
\caption{Biểu mẫu phiếu điều chỉnh kho}
\label{tab:bm14-phieu-dieu-chinh-kho}
\end{table}

\subsubsection{Quy định (QĐ)}

\textbf{QĐ5: Quy định về Kiểm kê kho}

\begin{itemize}
    \item Quản lý kho lập kế hoạch và tạo "Phiên Kiểm kê" trên hệ thống.
    \item Nhân viên kho bắt buộc sử dụng thiết bị di động để quét mã vạch/QR và nhập số lượng thực tế. Nghiêm cấm ước lượng hoặc nhập số sách để nhập.
    \item Hệ thống tự động đối chiếu và hiển thị các mã hàng bị chênh lệch ngay lập tức.
    \item Quản lý phải ra soát Báo cáo Chênh lệch và xác nhận "Phiếu Điều chỉnh Kho" để đưa tồn kho sổ sách về đúng với thực tế.
\end{itemize}

\end{document}