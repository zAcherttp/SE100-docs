\documentclass[../chapter1.tex]{subfiles}
\begin{document}

\subsection{Các quy mô và phần mềm hiện nay}

Thị trường phần mềm quản lý kho tại Việt Nam rất đa dạng, từ các giải pháp của những "ông lớn" toàn cầu như SAP EWM, Oracle NetSuite WMS đến các sản phẩm được phát triển bởi các công ty trong nước, phù hợp với đặc thù doanh nghiệp Việt Nam như KiotViet, Sapo, MISA AMIS, Viettel Post.

\begin{itemize}
    \item \textbf{Phân khúc hoạt động:} Thị trường có thể chia thành các phân khúc chính:
    \begin{itemize}
        \item \textbf{Doanh nghiệp lớn \& Tập đoàn đa quốc gia:} Thường lựa chọn các giải pháp ERP có module WMS mạnh mẽ như SAP, Oracle. Các doanh nghiệp này có quy mô hàng chục nghìn nhân viên, quản lý hàng trăm nghìn mã sản phẩm (SKU) tại nhiều kho lớn trên toàn quốc. Ví dụ: Vinamilk, Masan Consumer được biết đến là đang đầu tư mạnh vào SAP để quản lý chuỗi cung ứng phức tạp của mình.
        \item \textbf{Doanh nghiệp vừa và nhỏ (SMEs):} Đây là phân khúc sôi động nhất với sự tham gia của các nhà cung cấp trong nước. Các doanh nghiệp này có quy mô từ vài chục đến vài trăm nhân viên, quản lý một hoặc một vài kho. Họ cần các giải pháp linh hoạt, chi phí hợp lý và dễ triển khai.
        \item \textbf{Cửa hàng bán lẻ, Hộ kinh doanh cá thể:} Thường sử dụng các phần mềm quản lý bán hàng có tích hợp chức năng quản lý kho cơ bản. Đối tượng này tập trung vào các nghiệp vụ đơn giản như nhập, xuất, kiểm kho và báo cáo tồn kho.
    \end{itemize}
\end{itemize}

Dựa trên khảo sát thực tế, chúng ta có thể tổng kết các mức độ triển khai phần mềm quản lý kho tại Việt Nam như sau:

\begin{table}[H]
\centering
\begin{tabular}{|L{3cm}|L{4.5cm}|L{7cm}|}
\hline
\rowcolor{headerblue}
\textcolor{white}{\textbf{Quy mô doanh nghiệp}} & \textcolor{white}{\textbf{Ví dụ thực tế \& Triển khai}} & \textcolor{white}{\textbf{Số lượng người dùng \& Mức độ sử dụng}} \\
\hline
\textbf{Hộ kinh doanh, Cửa hàng nhỏ lẻ} & Các cửa hàng tạp hóa, shop thời trang online, quán cà phê. & \textbf{Đối tượng:} Chủ cửa hàng, 1-2 nhân viên bán hàng. \textbf{Mức độ:} Sử dụng các tính năng cơ bản trên phần mềm như KiotViet, Sapo POS. Chủ yếu dễ tập trung vào tồn kho, nhập hàng mới và bán hàng. Phần mềm thường là một module trong hệ thống quản lý POS. \\
\hline
\textbf{Doanh nghiệp vừa (SMEs) \& Chuỗi cửa hàng} & Chuỗi siêu thị mini (WinMart+), các công ty phân phối hàng tiêu dùng, doanh nghiệp sản xuất nhỏ. & \textbf{Đối tượng:} Kế toán kho, nhân viên kho, quản lý kho, ban lãnh đạo. \textbf{Mức độ:} Sử dụng các phần mềm chuyên sâu hơn như MISA, AMIS hoặc Odoo, Base WMS. Triển khai đầy đủ các nghiệp vụ từ đặt hàng, nhập, xuất, kiểm kê, báo cáo. Phần mềm thường có thể cần tích hợp với nhiều hệ phụ hợp với quy trình riêng của doanh nghiệp. \\
\hline
\textbf{Doanh nghiệp lớn, Tập đoàn, Logistics} & Các trung tâm phân phối lớn của Lazada, Shopee; các công ty logistics như GHN, J\&T; các nhà máy sản xuất lớn.. & \textbf{Đối tượng:} Hàng trăm nhân viên kho, quản lý các cấp, bộ phận chuỗi cung ứng, IT, ban lãnh đạo.  \textbf{Mức độ:} Triển khai các hệ thống WMS phức tạp (SAP EWM, Oracle WMS) tích hợp sâu với ERP, TMS (Hệ thống quản lý vận tải). Quản lý đa kho, tự động hóa cao với robot, băng chuyền. Phân quyền chi tiết cho từng vai trò. Có thể mở rộng cho các đối tác (3PL - Third-party logistics) cùng sử dụng trên một nền tảng. \\
\hline
\textbf{Mô hình Nhượng quyền (Franchise)} & Các chuỗi F\&B như Highlands Coffee, The Coffee House. & \textbf{Đối tượng:}  Cả công ty mẹ và các cửa hàng nhượng quyền. \textbf{Mức độ:} Công ty mẹ triển khai một hệ thống WMS/ERP tập trung. Các chi nhánh/cửa hàng nhượng quyền được cấp tài khoản để đặt hàng từ kho tổng, báo cáo tồn kho hàng ngày. Hệ thống giúp công ty mẹ quản lý toàn bộ chuỗi cung ứng, đảm bảo tính nhất quán về nguyên vật liệu và sản phẩm. \\
\hline
\end{tabular}
\caption{So sánh mức độ triển khai phần mềm quản lý kho theo quy mô}
\label{tab:phan-khuc-thi-truong}
\end{table}

Từ những phân tích trên, ta có thể rút ra các kết luận sau:

\begin{itemize}
    \item \textbf{Đối tượng sử dụng:} Phần mềm quản lý kho phục vụ một đối tượng rất rộng, từ cá nhân kinh doanh đến các tập đoàn đa quốc gia.
    \item \textbf{Quy mô triển khai:} Có vô số doanh nghiệp đang sử dụng các phần mềm tỷ lệ thuận với quy mô của doanh nghiệp. Doanh nghiệp càng lớn, quy trình càng phức tạp thì yêu cầu về một hệ thống WMS chuyên sâu, tùy biến cao.
    \item \textbf{Mức độ sử dụng:} Có nhiều mức độ sử dụng phần mềm:
    \begin{itemize}
        \item \textbf{Mức cơ bản:} Chỉ quản lý nhập - xuất - tồn.
        \item \textbf{Mức trung bình:} Quản lý theo lô/date, vị trí, kiểm kê, báo cáo chi tiết.
        \item \textbf{Mức nâng cao:} Tối ưu hóa lộ trình lấy hàng, cơ lý sắp xếp kho, tích hợp với thiết bị tự động (robot, xe tự hành), phân tích dữ liệu lớn (Big Data) để dự báo.
    \end{itemize}
    \item \textbf{Xu hướng phát triển:} Doanh nghiệp có xu hướng bắt đầu với các giải pháp đơn giản và nâng cấp dần khi quy mô mở rộng. Việc "lái lò đúc, cả liền cũng hạng" là lát yếu. Một phần mềm nghiệp vụ hoàn toàn có thể được thiết kế để phục vụ nhiều đối tượng kinh doanh khác nhau thông qua việc module hóa và cho phép tùy chỉnh linh hoạt.
\end{itemize}

\end{document}