\documentclass[../../main.tex]{subfiles}

\usepackage{array}
\usepackage{float}
\usepackage[table]{xcolor}
\usepackage{longtable}

% Define colors
\definecolor{headerblue}{RGB}{67, 113, 183}

% Custom column types
\newcolumntype{L}[1]{>{\raggedright\arraybackslash}p{#1}}
\newcolumntype{C}[1]{>{\centering\arraybackslash}p{#1}}

\begin{document}

\chapter{Tổng kết}

\section{Kết quả đạt được}

Sau quá trình nghiên cứu, khảo sát, phân tích yêu cầu và phát triển kéo dài trong suốt học kỳ, nhóm đã hoàn thành việc xây dựng hệ thống quản lý kho Next-WMS (Warehouse Management System). Đây là một hệ thống phần mềm toàn diện được thiết kế nhằm hỗ trợ các doanh nghiệp trong việc quản lý hoạt động kho bãi một cách hiệu quả, từ quy mô nhỏ đến vừa và lớn.

Xuất phát từ việc khảo sát thực trạng các giải pháp quản lý kho hiện có trên thị trường (đã trình bày chi tiết trong Chương 1), nhóm nhận thấy nhu cầu cấp thiết về một hệ thống quản lý kho với giao diện thân thiện, chi phí hợp lý, và khả năng tùy biến cao phù hợp với đặc thù của doanh nghiệp Việt Nam. Dựa trên cơ sở đó, hệ thống Next-WMS đã được phát triển với các mục tiêu cốt lõi: đơn giản hóa quy trình nghiệp vụ, tự động hóa các tác vụ lặp đi lặp lại, cung cấp thông tin chính xác theo thời gian thực, và hỗ trợ ra quyết định dựa trên dữ liệu.

\subsection{Tổng quan kết quả}

Về mặt chức năng, hệ thống đã triển khai thành công \textbf{100 use case} được phân bổ vào \textbf{11 nhóm chức năng chính} như đã mô tả chi tiết trong Chương 2. Các nhóm chức năng này bao phủ toàn bộ vòng đời quản lý kho, từ thiết lập hệ thống ban đầu, quản lý dữ liệu nền tảng, cho đến các nghiệp vụ cốt lõi như nhập kho, xuất kho, kiểm kê, điều chuyển nội bộ, và báo cáo phân tích.

Về mặt kỹ thuật, hệ thống được xây dựng trên nền tảng công nghệ hiện đại bao gồm:
\begin{itemize}
    \item \textbf{Frontend:} Next.js 14 với App Router, React 18, TypeScript, và thư viện giao diện shadcn/ui
    \item \textbf{Backend:} Convex - nền tảng Backend-as-a-Service với real-time database và serverless functions
    \item \textbf{Authentication:} Better Auth với hỗ trợ xác thực đa phương thức (email/password, OTP)
    \item \textbf{Database:} Convex Database (NoSQL) kết hợp với Neon PostgreSQL cho authentication
    \item \textbf{Deployment:} Vercel (frontend) và Convex Cloud (backend)
\end{itemize}

\subsection{Các thành quả nổi bật}

Trong quá trình phát triển, nhóm đã đạt được các thành quả đáng chú ý sau:

\begin{itemize}
    \item \textbf{Xây dựng thành công kiến trúc đa tổ chức (Multi-tenant):} Hệ thống cho phép nhiều tổ chức (doanh nghiệp) hoạt động độc lập trên cùng một nền tảng. Mỗi tổ chức có không gian làm việc (workspace) riêng biệt với dữ liệu được cô lập hoàn toàn, đảm bảo tính bảo mật và riêng tư. Một tổ chức có thể quản lý nhiều chi nhánh kho với cấu hình và quyền hạn riêng.
    
    \item \textbf{Triển khai hệ thống phân quyền chi tiết:} Hệ thống RBAC (Role-Based Access Control) được thiết kế với 3 cấp độ cơ bản (Owner, Admin, Member) và hỗ trợ tạo vai trò tùy chỉnh. Quyền hạn được kiểm soát ở mức chi tiết thông qua permission bits, cho phép cấu hình linh hoạt theo nhu cầu từng tổ chức.
    
    \item \textbf{Cập nhật dữ liệu thời gian thực:} Nhờ tích hợp Convex với WebSocket subscriptions, mọi thay đổi dữ liệu được đồng bộ ngay lập tức đến tất cả người dùng đang trực tuyến mà không cần refresh trang. Điều này đặc biệt quan trọng trong môi trường kho nơi nhiều nhân viên cùng làm việc đồng thời.
    
    \item \textbf{Hỗ trợ theo dõi hàng hóa đa dạng:} Hệ thống hỗ trợ theo dõi hàng hóa theo nhiều phương thức: theo số lượng đơn thuần, theo lô hàng (batch tracking) với ngày sản xuất và hạn sử dụng, hoặc theo số sê-ri (serial number) cho các sản phẩm có giá trị cao cần truy xuất nguồn gốc.
    
    \item \textbf{Số hóa toàn bộ quy trình nghiệp vụ:} Các quy trình nghiệp vụ như nhập kho, xuất kho, kiểm kê, điều chuyển đều được số hóa với workflow rõ ràng, bao gồm các trạng thái, bước phê duyệt, và ghi nhận lịch sử thay đổi.
    
    \item \textbf{Hệ thống thông báo và cảnh báo thông minh:} Hệ thống tự động phát hiện và cảnh báo các vấn đề quan trọng như hàng sắp hết hạn, tồn kho dưới mức an toàn, các yêu cầu chờ phê duyệt, giúp người dùng chủ động xử lý kịp thời.
    
    \item \textbf{Ghi nhật ký hoạt động (Audit Log) đầy đủ:} Mọi thao tác trên hệ thống đều được ghi lại chi tiết, bao gồm người thực hiện, thời gian, giá trị trước và sau khi thay đổi, phục vụ cho việc kiểm tra, đối soát và tuân thủ quy định.
    
    \item \textbf{Thiết kế cơ sở dữ liệu chuẩn hóa:} Cơ sở dữ liệu được thiết kế với 23 bảng chính, đảm bảo tính toàn vẹn dữ liệu, hỗ trợ mở rộng trong tương lai, và tối ưu cho các truy vấn phổ biến.
    
    \item \textbf{Giao diện người dùng hiện đại và thân thiện:} Giao diện được xây dựng theo nguyên tắc thiết kế hiện đại, hỗ trợ responsive trên nhiều kích thước màn hình, có chế độ sáng/tối (dark mode), và đảm bảo khả năng tiếp cận (accessibility).
\end{itemize}

\subsection{Thông tin triển khai}

\begin{itemize}
    \item \textbf{Mã nguồn:} \url{https://github.com/zAcherttp/SE100-docs.git}
    \item \textbf{Video demo:} [Link video demo - Đã bật quyền chia sẻ]
    \item \textbf{Người quản lý mã nguồn:} [Tên sinh viên - MSSV]
    \item \textbf{Người quản lý video demo:} [Tên sinh viên - MSSV]
\end{itemize}

\section{Ưu và khuyết điểm của phần mềm}

\subsection{Ưu điểm}

\begin{itemize}
    \item \textbf{Giao diện hiện đại với shadcn/ui:} Sử dụng shadcn/ui dựa trên Radix UI, đảm bảo accessibility (a11y) và có thể tùy chỉnh cao. Hỗ trợ dark mode và responsive design.
    
    \item \textbf{Realtime updates với Convex:} Mọi thay đổi dữ liệu được đồng bộ realtime qua WebSocket, người dùng thấy cập nhật ngay lập tức mà không cần refresh.
    
    \item \textbf{Type-safe toàn bộ stack:} Sử dụng TypeScript từ frontend đến backend với Convex schema validation, giảm thiểu lỗi runtime.
    
    \item \textbf{Kiến trúc Monorepo với Turborepo:} Quản lý code hiệu quả, shared packages, và incremental builds giúp tăng tốc độ phát triển.
    
    \item \textbf{Quản lý đa tổ chức (Multi-tenant):} Hỗ trợ nhiều organization với dữ liệu được cô lập, phù hợp cho SaaS.
    
    \item \textbf{Phân quyền chi tiết:} Hệ thống RBAC với permission bits cho phép kiểm soát quyền truy cập ở mức chi tiết.
    
    \item \textbf{Theo dõi lô hàng và serial:} Hỗ trợ batch tracking và serial number tracking, phù hợp cho ngành dược phẩm, thực phẩm.
    
    \item \textbf{Audit logging đầy đủ:} Ghi nhận mọi thay đổi với old/new value, hỗ trợ compliance và troubleshooting.
    
    \item \textbf{Cảnh báo tự động:} Scheduled jobs kiểm tra hàng sắp hết hạn và gửi thông báo tự động.
    
    \item \textbf{Serverless architecture:} Không cần quản lý server, tự động scale với Convex và Vercel.
\end{itemize}

\subsection{Khuyết điểm}

\begin{itemize}
    \item \textbf{Chưa hỗ trợ quét mã vạch tích hợp:} Tính năng quét mã vạch qua camera chưa được triển khai, cần phát triển thêm để tối ưu quy trình nhập xuất kho.
    
    \item \textbf{Thiếu module dự báo nhu cầu:} Bảng demand\_forecasts đã được thiết kế nhưng chưa triển khai logic dự báo dựa trên dữ liệu lịch sử.
    
    \item \textbf{Chưa tích hợp API bên ngoài:} Chưa có public API cho phép tích hợp với hệ thống ERP, kế toán hoặc e-commerce platforms.
    
    \item \textbf{Thiếu giao diện 2D warehouse layout:} Chưa có tính năng visualize sơ đồ kho với drag-and-drop zone management.
    
    \item \textbf{Phụ thuộc vào Convex:} Sử dụng BaaS (Backend-as-a-Service) tạo vendor lock-in, khó migrate sang platform khác.
    
    \item \textbf{Chưa hỗ trợ offline mode:} Ứng dụng yêu cầu kết nối internet liên tục, chưa có PWA offline capabilities.
    
    \item \textbf{Tài liệu người dùng chưa đầy đủ:} Chỉ có documentation kỹ thuật, thiếu user manual cho end users.
    
    \item \textbf{Chưa có unit tests đầy đủ:} Test coverage còn hạn chế, cần bổ sung tests cho critical functions.
\end{itemize}

\section{Hướng phát triển}

Dựa trên các khuyết điểm hiện tại và nhu cầu thực tế, hệ thống có thể được phát triển thêm theo các hướng sau:

\subsection{Ngắn hạn (3-6 tháng)}

\begin{itemize}
    \item \textbf{Tích hợp quét mã vạch:} Sử dụng thư viện như html5-qrcode hoặc @scandit/websdk để quét barcode trực tiếp trong browser.
    
    \item \textbf{Bổ sung unit tests:} Viết tests cho Convex functions với convex-test, đạt coverage tối thiểu 80\%.
    
    \item \textbf{Tối ưu performance:} Implement pagination cho các list queries, sử dụng Convex indexes hiệu quả hơn.
    
    \item \textbf{Hoàn thiện Reports module:} Bổ sung các báo cáo chi tiết với export Excel/PDF.
    
    \item \textbf{User documentation:} Viết hướng dẫn sử dụng với screenshots và video tutorials.
\end{itemize}

\subsection{Trung hạn (6-12 tháng)}

\begin{itemize}
    \item \textbf{Module dự báo nhu cầu:} Triển khai demand forecasting sử dụng historical data và simple moving average.
    
    \item \textbf{2D Warehouse Layout:} Phát triển interactive warehouse map với React Flow hoặc Konva.js.
    
    \item \textbf{Public REST API:} Xây dựng API layer với Next.js API routes cho third-party integrations.
    
    \item \textbf{Mobile App với React Native:} Phát triển app mobile sử dụng React Native với shared logic từ web.
    
    \item \textbf{Advanced Analytics:} Tích hợp ABC analysis, inventory turnover analysis, và custom dashboards.
    
    \item \textbf{Email reports:} Scheduled email reports hàng ngày/tuần với React Email templates.
\end{itemize}

\subsection{Dài hạn (1-2 năm)}

\begin{itemize}
    \item \textbf{AI-powered features:}
    \begin{itemize}
        \item Gợi ý vị trí lưu trữ tối ưu dựa trên AI
        \item Anomaly detection cho inventory discrepancies
        \item Demand forecasting với machine learning
    \end{itemize}
    
    \item \textbf{IoT Integration:}
    \begin{itemize}
        \item Tích hợp cảm biến nhiệt độ, độ ẩm qua MQTT
        \item RFID reader integration cho asset tracking
        \item Cân điện tử kết nối cho weight-based inventory
    \end{itemize}
    
    \item \textbf{PWA với Offline Support:}
    \begin{itemize}
        \item Service Worker cho offline caching
        \item Background sync khi có mạng trở lại
        \item IndexedDB cho local data storage
    \end{itemize}
    
    \item \textbf{Route Optimization:} Thuật toán tối ưu picking path trong warehouse.
    
    \item \textbf{Multi-language Support:} i18n với next-intl, hỗ trợ tiếng Việt và tiếng Anh.
    
    \item \textbf{Advanced Security:}
    \begin{itemize}
        \item Two-factor authentication (2FA)
        \item IP whitelisting
        \item Session management nâng cao
    \end{itemize}
\end{itemize}

\subsection{Kết luận}

Hệ thống quản lý kho Next-WMS đã được xây dựng thành công với kiến trúc hiện đại sử dụng Next.js, Convex, và Better Auth. Hệ thống đáp ứng đầy đủ các chức năng cơ bản và nâng cao cho quản lý kho doanh nghiệp, bao gồm: quản lý sản phẩm đa dạng với variants và barcodes, theo dõi tồn kho realtime với batch tracking, quy trình nhận hàng và cycle count có hệ thống, phân quyền chi tiết theo role và branch, cùng hệ thống cảnh báo tự động.

Với nền tảng công nghệ serverless và type-safe, hệ thống có khả năng mở rộng tốt và dễ bảo trì. Tuy còn một số hạn chế như thiếu offline mode và barcode scanning, nhưng với lộ trình phát triển rõ ràng, Next-WMS có tiềm năng trở thành một giải pháp quản lý kho toàn diện, cạnh tranh được với các sản phẩm thương mại trên thị trường.

\end{document}