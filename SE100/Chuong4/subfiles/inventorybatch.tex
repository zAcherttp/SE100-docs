\documentclass[../chapter4_clean.tex]{subfiles}
\begin{document}
\subsection{InventoryBatch}

\begin{longtable}{|C{0.8cm}|L{4cm}|L{3cm}|L{2.5cm}|L{4cm}|}
\hline
\rowcolor{headerblue}
\textcolor{white}{\textbf{.No}} & 
\textcolor{white}{\textbf{Attribute name}} & 
\textcolor{white}{\textbf{Data type}} & 
\textcolor{white}{\textbf{Constraint}} & 
\textcolor{white}{\textbf{Meaning or Note}} \\
\hline
\endfirsthead

\hline
\rowcolor{headerblue}
\textcolor{white}{\textbf{.No}} & 
\textcolor{white}{\textbf{Attribute name}} & 
\textcolor{white}{\textbf{Data type}} & 
\textcolor{white}{\textbf{Constraint}} & 
\textcolor{white}{\textbf{Meaning or Note}} \\
\hline
\endhead

1 & id & int & PK & Mã định danh lô hàng \\
\hline
2 & quantity & int & & Số lượng tồn kho \\
\hline
3 & supplierBatchNumber & string & & Số lô của nhà cung cấp \\
\hline
4 & internalBatchNumber & string & & Số lô nội bộ \\
\hline
5 & receivedAt & date & & Ngày nhận hàng \\
\hline
6 & expiresAt & date & & Ngày hết hạn \\
\hline
7 & status & BatchStatus & & Trạng thái lô hàng \\
\hline
\end{longtable}

\textbf{Methods:}
\begin{itemize}
    \item \texttt{adjustQuantity(change, reason)}: Điều chỉnh số lượng
    \item \texttt{checkExpiration()}: Kiểm tra hạn sử dụng
    \item \texttt{isExpired()}: Kiểm tra đã hết hạn
    \item \texttt{getDaysUntilExpiry()}: Lấy số ngày còn lại đến hết hạn
\end{itemize}

\end{document}
